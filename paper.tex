\documentclass[10pt]{sigplanconf}

\usepackage{amssymb}
\usepackage{amsmath}
\usepackage{amsfonts}
\usepackage{graphicx}
\usepackage{url}
\usepackage{xspace}
\usepackage{listings}
\usepackage[noline]{algorithm2e}
\usepackage{comment}
\usepackage{color}
\usepackage{times}
\usepackage{multirow}
\usepackage{balance}
\usepackage{framed}
\usepackage{alltt}
\usepackage{textcomp}
\usepackage[dvipsnames]{xcolor}

\setcounter{tocdepth}{3}


% configures style of source code listings
\lstset{
	basicstyle=\small\sffamily
	%basicstyle=\scriptsize\ttfamily
	%numbers=left
}

\lstdefinelanguage{java}{
	morekeywords={abstract,continue,for,new,switch,assert,default,goto,package,synchronized,boolean,stream,event,filter,when,do,if,private,this,break,double,implements,protected,throw,byte,else,import,public,throws,case,enum,instanceof,return,transient,catch,extends,int,short,try,char,final,interface,static,void,class,finally,long,strictfp,volatile,const,float,native,super,while},
	sensitive=false,
	morestring=[b]",
	morestring=[b]',
	morecomment=[l]{//},
	morecomment=[s]{/*}{*/)},
	escapeinside=??,
	moredelim=[is][\textit]{__}{__},
	moredelim=[is][\textbf]{_*}{*_}
}
\lstnewenvironment{displayjava}
	{\lstset{language=java,basicstyle=\small\sffamily,tabsize=2,columns=fullflexible,captionpos=b,xleftmargin=1em,xrightmargin=1em}}{}
\lstnewenvironment{java}
	{\lstset{language=java,basicstyle=\small\sffamily,tabsize=2,columns=fullflexible,captionpos=b}}{}


\newcommand{\code}[1]{{\sf \small #1}\xspace}

\newcommand{\mi}[1]{\mathit{#1}}
\newcommand{\mr}[1]{\mathrm{#1}}
\newcommand{\mt}[1]{\mathtt{#1}}

\newcommand{\todo}[1]{{\bfseries TODO #1}}

% for writing comments
\newcommand{\PP}[1]{\textcolor{blue}{\bfseries PP: #1}} % Pavel Parizek
\newcommand{\PV}[1]{\textcolor{red}{\bfseries PV: #1}} % Premek Vysoky
\newcommand{\JB}[1]{\textcolor{orange}{\bfseries JB: #1}} % JetBrains

% coloring and font style commands
  % ANTLR
\newcommand{\antlrap}{\textquotesingle}
\newcommand{\antlrparserrule}[1]{\textcolor{Blue}{#1}}
\newcommand{\antlrlexerrule}[1]{\textcolor{RedOrange}{#1}}
\newcommand{\antlrliteralnoap}[1]{\textcolor{Maroon}{#1}}
\newcommand{\antlrliteral}[1]{\antlrap\antlrliteralnoap{#1}\antlrap}
\newcommand{\antlrregex}[1]{\textcolor{PineGreen}{#1}}
  % MPS in general
\newcommand{\mpsconcept}[1]{\textcolor{OliveGreen}{#1}}
\newcommand{\mpsinterface}[1]{\textcolor{Plum}{#1}}
  % MPS structure
\newcommand{\mpsstkeyword}[1]{\textcolor{Blue}{\textbf{#1}}}
\newcommand{\mpsstalias}[1]{\textcolor{Green}{\textbf{#1}}}
\newcommand{\mpsstproperty}[1]{\textcolor{Fuchsia}{\textbf{#1}}}
\newcommand{\mpsstplaceholder}[1]{\textcolor{Gray}{\textbf{#1}}}
\newcommand{\mpsstcardinality}[1]{\textcolor{RoyalBlue}{#1}}
  % MPS editor
\newcommand{\mpsedkeyword}[1]{\textcolor{Black}{\textbf{#1}}}
\newcommand{\mpsedtarget}[1]{\textcolor{Fuchsia}{#1}}
\newcommand{\mpsedannotation}[1]{\textcolor{Gray}{\textbf{#1}}}
  % MPS textgen
\newcommand{\mpstgkeyword}[1]{\textcolor{Blue}{\textbf{#1}}}
\newcommand{\mpstgtarget}[1]{\textcolor{Fuchsia}{#1}}
\newcommand{\mpstgparam}[1]{\textcolor{Violet}{#1}}
\newcommand{\mpstgaction}[1]{\textcolor{Fuchsia}{\textbf{#1}}}
\newcommand{\mpstgliteral}[1]{\textcolor{Green}{\textbf{#1}}}
\newcommand{\mpstgnodeprop}[1]{\textcolor{darkgray}{#1}}



\begin{document}


% TODO
%\conferenceinfo{PLDI'17,} {June 19--26, 2017, Barcelona, Spain.}
%\CopyrightYear{2017}
%\copyrightdata{TODO}


\title{INGRID: Creating Languages in MPS from ANTLR Grammars}


\authorinfo{P\v{r}emysl Vysok\'y \and Pavel Par\'izek}
	{Charles University, Faculty of Mathematics and Physics, Prague, Czech Republic}
	{premek.vysoky@gmail.com, parizek@d3s.mff.cuni.cz}

% TODO pridat nekoho dalsiho z JetBrains (Alex Shatalin?)
% TODO upravit affiliation pro JetBrains -> doplnit "s.r.o" nebo radsi "Inc."
% TODO muzeme taky napsat jejich emaily pokud budou lidi z JetBrains souhlasit
\authorinfo{V\'aclav Pech}
	{JetBrains}


\maketitle

\begin{abstract}
% TODO
\todo{write later}

\todo{Copy from master thesis.}
JetBrains MPS is a language workbench focusing on domain-specific languages.
Unlike many other language workbenches and IDEs, it uses a projectional editor for code.
The developer directly manipulates the program in its tree form (AST) and not by editing a text source code.
This brings many advantages, but on the other hand requires time-consuming and complicated MPS language definition.
The thesis elaborates on the possibility of automating the process of creating MPS language definition from its grammar description.
It introduces the MPS editor, evaluates approaches of related projects and describes author's efforts to implement an MPS plugin that allows this import.
The chosen approach and the selection of tools used for implementation are justified in the thesis.
We point out important problems that any similar project might deal with and we introduce some possible solutions.
Furthermore, the thesis contains examples of imported languages, showing the potency of the chosen approach.
The thesis also aims to lay groundwork for future extensions and suggest possible improvements.
\end{abstract}

% TODO possible keywords: JetBrains MPS, programming language, grammar, projectional editor


\section{Introduction}

The prevailing approach to define valid syntaxes for programming languages is through grammars written in notations based on the Extended Backus-Naur Form (EBNF).
Many tools exist for automated generation of parsers from the grammar definitions --- for example, ANTLR~\cite{ref:ANTLRBOOK,ref:ANTLR} and Bison/Flex~\cite{ref:BISONFLEX}.
The source code of programs is written in a text editor, while parsers are used to create an abstract syntax tree (AST) representation of the code.

An alternative approach is projectional (or structural) editing~\cite{ref:VWK15,ref:VSB14,ref:DHL80}, where a developer directly manipulates the AST representation of the source code instead of plain text.
Recently, projectional editing has become widely used in the area of language workbenches --- IDE-like tools that enable the developers to manipulate language definitions.
Popular examples are JetBrains MPS~\cite{ref:MPS,ref:MPSBOOK} and Spoofax~\cite{ref:KV10}.
In particular, JetBrains MPS is an open-source language workbench that focuses on domain specific languages (DSL) and leverages the concept of projectional editing.
MPS provides the whole IDE infrastructure that enables developers to design custom languages, use them to write program code, and compile the code into executables.

Projectional editing, keeping the code in the AST form, and the absence of parsers, brings along several benefits.
\begin{itemize}
	\item A projectional editor does not allow the developers to enter syntactically invalid code.
	\item Programming languages can be defined in a modular way, and multiple languages can be easily combined together in a single program or one can extend another.
	\item The languages may support diverse contextual or non-parseable notations with complex layout, such as tables, diagrams, and mathematical expressions.
	\item Since the projection is detached from the physical representation of code (AST), authors of languages can define multiple notations and allow the developers to switch between them on the screen.
\end{itemize}
All of this is useful for DSLs, which are often used by domain experts who may not be professional software developers.

Nevertheless, the usage of projectional editors introduces some new problems.
Before a language can be used inside an editor, it has to be defined through the specific infrastructure, so that the IDE can understand the language and work with it.
More precisely, an author of a language has to create:
\begin{enumerate}
	\item the abstract syntax (structure of the language), which defines the types of allowed AST nodes,
	\item the concrete syntax for editing, i.e. projection of the AST on the screen and interactions with the user, and
	\item text generation scripts to enable creation of plain text representation used as input for compilers.
\end{enumerate}	

IDE tools based on projectional editors are typically used for syntactically rich DSLs.
In the specific case of MPS, before a program written in a DSL can be executed, it is transformed on the AST level by a series of model-to-model transformations to an AST model that represents the desired semantics in a general-purpose programming language (GPPL), such as C or Java.
This model is then converted to a textual representation of that language and compiled by the standard means of the target platform.
Therefore, the target GPPL must also be defined in MPS, using the respective infrastructure.

However, very few mainstream GPPLs are now supported by MPS, because it requires substantial effort to manually create a full definition of a language in MPS.
Only Java and C have been implemented to date.
The overall goal of our project is to automatize the process of language definition in MPS as much as possible.
Greater automation would encourage and speed-up the migration of more GPPLs into MPS, thus giving the authors of DSLs more options regarding target general languages.
Another possible application is the combination of general-purpose languages with DSLs natively created in MPS.

In this paper we present \textsc{Ingrid}, a method and a tool for semi-automated construction of a language definition in MPS that uses an ANTLR grammar as a description of its syntax.
The process of construction is semi-automated for two reasons: (1) ANTLR grammars have to be adjusted before \textsc{Ingrid} can process them, and (2) some aspects of the language definitions in MPS typically must be tweaked or created manually in order to improve languages' usability.

We also discuss the main challenges that we encountered during the development of \textsc{Ingrid} and our solution to them.
They are related to principal differences between the two approaches to the definition of a programming language --- (i) one that uses grammars in an EBNF-like notation, with rules and tokens written in plain text, and parser generators, and (ii) the structured object-oriented approach used in MPS and other language workbenches.
Specifically, the description of a language in the form of a grammar does not hold any information about the code layout, and the grammar typically contains many rules that do not directly correspond to AST nodes and programming language constructs.

\textsc{Ingrid} can also make the construction of DSLs more efficient.
Our experience shows that, especially for simple languages, it is less time consuming to write their grammar by hand and then create the language definition in MPS with the help of \textsc{Ingrid}, than to create everything manually using the MPS GUI.

We make the following key contributions:
\begin{itemize}
	\item The \textsc{Ingrid} method and tool for construction of MPS language definitions from ANTLR grammars, which can be used (1) for the import of general-purpose languages into MPS and (2) for efficient construction of DSLs. \textsc{Ingrid} can be adapted also to other language workbenches based on similar principles as MPS.
	\item Identification and discussion of general challenges related to the differences between (1) language definitions that use grammars and (2) the structured object-oriented approach used in projectional editors.
	\item Discussion of the practical usability of generated MPS languages and its dependence on (1) the specific form of ANTLR grammars and (2) the manual adjustments of input grammars and specific aspects of MPS languages (e.g., with respect to code layout).
\end{itemize}

In the next section, we provide (i) the necessary information about the MPS infrastructure for language definition and (ii) a brief overview of projects with similar goals as \textsc{Ingrid}.
Then, in Section~\ref{sect:LANGDEF}, we describe our approach to the construction of a language definition in MPS from its ANTLR grammar, discussing the major challenges together with our solution.
We also discuss our experience with application of \textsc{Ingrid} on complex mainstream languages, such as JavaScript and C\# (Section~\ref{sect:EVAL}), and then we outline our plans for the future work.

Note also that we omitted many important and interesting details from this paper because of limited space.
Readers can find the additional details in the full version~\cite{ref:TRFULL}.


\section{Background and Related Work}
\label{sect:BACKGRELWORK}

In this section, first we introduce basic features of the MPS platform, and then we discuss other existing projects that aim to add support for general purpose languages into MPS.

\subsection{JetBrains MPS}
\label{sect:MPS}

JetBrains MPS is a complete language workbench --- an integrated development environment that allows developers to create their own languages and use them to write code.
The code can then be transformed into a target language, typically a GPL, such as Java or C, and eventually compiled into executable programs.

As we already indicated, MPS differs from typical IDEs in one important aspect --- the projectional editor.
The developer does not work with the textual representation of the source code directly, but rather with its AST that is the model of the code.
Basically, when using the MPS projectional editor, programs are created by assembling the tree (AST) out of predefined building blocks of selected languages.
The definition of a language in MPS dictates, where in the AST can certain elements be placed and how they can be nested inside other elements.
On the other hand, in traditional IDEs it is the parser that constructs an AST of a program using the language's elements.

The building blocks of MPS models are called nodes.
Code of any program in MPS is built from nodes, which represent instances of concepts defined in one of the languages that the program declares to be using.
In the context of MPS, a concept is, in fact, a building block of a language definition.
We use the terms \emph{MPS concept} and \emph{AST node} when needed to avoid confusion.

One of the key advantages of projectional editing stems from the separation of abstract and concrete syntax.
While AST provides a complete and precise representation of the code, the way it is displayed on the screen and the way the user interacts with it are unconstrained.
The editor can take any visual form and shape.
The language author can define multiple alternative visualizations and let the developer to choose one that fits best the task at hand --- for example, editing, debugging, reviewing, resolving merge conflicts, etc.
In particular, the visual representations are not bound to be just textual at all.
For example, blocks corresponding to branches of an \code{if-then-else} statement can be aligned next to each other and displayed with different background colors.

The definition of an MPS concept (language element) consists of several aspects, where each aspect codifies a different part of the AST nodes' behavior.
The essential aspects are the following: \emph{Structure} (abstract syntax), \emph{Editor} (concrete syntax) and \emph{TextGen}.
Editor defines the concrete syntax (i.e., how the code is visualized and edited) and TextGen specifies how AST nodes are transformed into textual representation.
If, instead of generating text directly, programs in the language are supposed to be transformed into another language that is available in MPS, the Generator aspect must be used to specify the model-to-model transformation rules.
Since only the Structure, Editor and TextGen aspects are relevant for the contribution of this paper, we describe these three below in more detail, and neglect other aspects such as type system and data-flow.

Languages are built from the concepts using techniques known from object-oriented programming --- containment, inheritance, interfaces, and so on.
Therefore, a definition of a whole language in MPS typically has an object-oriented and hierarchical nature.

\paragraph{Structure.}
The fundamental aspect of any MPS language is Structure.
It must be created first for each intended element (concept) of the language.
Structure specifies core attributes of an MPS concept such as the name, inheritance relationships, possible child concepts (including their types and cardinalities), the implemented interfaces, and references to other AST nodes.
% other properties (fields of any type that can hold values), etc.
Figure~\ref{fig:if_statement_structure} shows definition of the Structure aspect for the \code{if-else} statement.

The definition of Structure restricts the type of AST nodes that can appear at a particular place in the tree.
For example, one can restrict the condition in the \code{if-then-else} statement to be a boolean expression, and the \code{then}-block to be a list of statements.

\begin{figure}[ht]
\centering
\begin{alltt}
\small
\mpsstkeyword{concept} IfStatement \mpsstkeyword{extends} Statement
        \mpsstkeyword{implements} IContainsStatementList
                   IDontSubstituteByDefault
                   IConditional

  \mpsstkeyword{instance can be root:} false
  \mpsstkeyword{alias:} \mpsstalias{if}
  \mpsstkeyword{short description:} \mpsstplaceholder{<no short description>}

  \mpsstkeyword{properties:}
  \mpsstproperty{forceOneLine}   : boolean
  \mpsstproperty{forceMultiLine} : boolean
  
  \mpsstkeyword{children:}
  \mpsstproperty{condition}        : Expression[\mpsstcardinality{1}]
  \mpsstproperty{ifFalseStatement} : Statement[\mpsstcardinality{0..1}]
  \mpsstproperty{ifTrue}           : StatementList[\mpsstcardinality{1}]
  \mpsstproperty{elsifClauses}     : ElsifClause[\mpsstcardinality{0..n}]
  
  \mpsstkeyword{references:}
  \mpsstplaceholder{<< ... >>}
\end{alltt}
\caption{Structure aspect of the \code{if-else} statement}
\label{fig:if_statement_structure}
\end{figure}

\paragraph{Editor.}
The Editor aspect is where the language designer specifies what the projectional editor representation of a code fragment (an AST) looks like on the screen and how the user interacts with the code.
JetBrains have developed a cellular system that enables placing properties and children of a node (concept) into different cells.
The author usually incorporates all of the node's children, references, and properties inside the representation, so that the future users of the language can insert all values that the node expects.
Additionally, cells of the editor can be styled using a language similar to CSS.
The supported visual characteristics include text color and indentation.

Figure~\ref{fig:if_editor_definition} shows an example of what the definition of the Editor aspect for the \code{if-else} statement might look like.
While here we indicate positions of cell borders on each line by spaces (just for the purpose of illustration), MPS GUI actually uses a graphically much more appealing way of displaying the Editor aspects, which involves vertical and horizontal lines of different colors and also background colors other than white for some cells.
The symbols \verb|[-| and \verb|-]| represent layout information (cells), such as indentation.

\begin{figure}[ht]
\centering
\begin{alltt}
\small
\mpsedannotation{<default>} \mpsedkeyword{editor for concept} \mpsedtarget{IfStatement}
  \mpsedkeyword{node cell layout:}
    [-
      \mpsedkeyword{if} \textbf{(} \% conditions \% \textbf{)} [-
      \textbf{\{}
      [- \% ifTrue % -]
      \textbf{\}}
    -]
    ?[- \mpsedkeyword{else} \% ifFalseStatement \% -]
    -]
\end{alltt}
\caption{Editor aspect for the \code{if-else} statement}
\label{fig:if_editor_definition}
\end{figure}

\paragraph{BaseLanguage.}
Another important feature of MPS that we need to describe in more detail is \emph{BaseLanguage}.
It is a clone of Java implemented using the MPS constructs.
BaseLanguage was developed in the early days of MPS in order to implement MPS itself and also to support the basic set of language-definition DSLs,
Although BaseLanguage is syntactically almost identical to Java, it is edited in a projectional editor, just like all the languages in MPS.
The language-definition DSLs, used to define custom languages by their authors, are generated into the BaseLanguage.
Similarly, all the custom DSLs that are meant to be generated into Java choose BaseLanguage as their generation target.
The conversion to textual Java is handled by BaseLanguage without any further manual effort, since BaseLanguage has a TextGen aspect defined, which translates BaseLanguage code into textual Java sources.

\paragraph{TextGen.}
The TextGen aspect specifies how a given AST node will be translated into plain text representation.
It is typically needed only for the bottom-line base languages.
DSLs, on the other hand, need to define rules for model-to-model conversions (Generators), since these are rarely converted to text directly.
After TextGen has generated textual sources from an AST, a compiler for the particular GPL is invoked to compile the textual sources into binary code.
The TextGen definition follows a very straightforward pattern --- each node outputs its textual representation into a buffer, calling TextGen of its children nodes at the right moments.
MPS calls the corresponding method on the root AST nodes of the given program.

TextGen aspect for each AST node (concept of the language) has to be defined using the BaseLanguage.
Again, we include an example for the \code{if-else} statement (Figure~\ref{fig:if_statement_textgen}).

\begin{figure}[ht]
\centering
\begin{alltt}
\small
\mpstgkeyword{text gen component for concept} \mpstgtarget{IfStatement} \{
  \mpstgparam{(context, buffer, node)->void} \{
    \mpstgaction{append} \textcolor{Blue}{\textbf{\textbackslash{}n;}}
    \mpstgaction{indent buffer;}
    \mpstgaction{append} \{\mpstgliteral{if (}\} \$\{\mpstgparam{node}.\mpstgnodeprop{condition}\} \{\mpstgliteral{) \{}\};
    \mpstgkeyword{with indent} \{
      \mpstgaction{append} \$\{\mpstgparam{node}.\mpstgnodeprop{ifTrue}\};
    \}
    \mpstgaction{append} \textcolor{Blue}{\textbf{\textbackslash{}n}} \{\mpstgliteral{\}}\} \$list\{\mpstgparam{node}.\mpstgnodeprop{elsifClauses}\};
    \mpstgkeyword{if} (\mpstgparam{node}.\mpstgnodeprop{ifFalseStatement}.\textbf{isNotNull}) \{
      \mpstgaction{append} \{ \mpstgliteral{else}\} \$\{\mpstgparam{node}.\mpstgnodeprop{isFalseStatement}\};
    \}
  \}
\}
\end{alltt}
\caption{TextGen aspect definition for the \code{if-else} statement}
\label{fig:if_statement_textgen}
\end{figure}





% TODO continue from here



% TODO mozna zmenit nazev teto sekce
\subsection{Similar Projects}

We are aware of several projects that provide certain support for GPLs in the context of the MPS platform.
We will analyze existing related projects that are trying to accomplish similar goals, and then we will weigh advantages and disadvantages of these approaches.

Currently, there exists an almost full port of the Java language called BaseLanguage~\cite{ref:BaseLanguage} extended with some MPS specific features.
It was imported manually by JetBrains and it is still undergoing changes as Java itself is evolving.

Within the mbeddr project~\cite{ref:mbeddr}, the C language is also manually tailored for MPS.

Now/here we describe three similar projects (with similar goals) that we are aware of.
We focus on (consider) these three projects: PE4MPS, ANTLR{\_}MPS, and mps-metabnf.

\paragraph{PE4MPS.}
PE4MPS~\cite{ref:PE4MPS} is a project by Marco Lombardo that is trying to solve the same problem as we do.
It solves the lack of information about code layout in grammars by creating a new grammar notation called PE Grammars~\cite{ref:PE} (the PE abbreviation comes from projectional editing).
It has tw components: PE parser and PE4MPS plugin for MPS.
Their approach is to mimic an existing grammar notation called ANTLRv4~\cite{ANTLR4} and enrich its syntax with custom (their own) constructs.
These extensions tell (guide, hint) the parser what the AST node layout should look like.
Supported extensions (already implemented) include horizontal and vertical lists, and some indentation rules --- however, even these few features make the already not-so-simple syntax of ANTLR v4 much more complex.
The parser then uses this information when generating the projectional editor for am AST node.
Author of this approach/project described the PE syntax using the ANTLRv4 notation~\cite{ANTLR4reference} and then used the ANTLR toolset to automatically generate an ANTLRv4 parser for PE grammar files.
The PE parser reads any PE file and stores the language structure found inside to a custom representation of Java objects.

The PE4MPS plugin/project/tool is built on top of the PE parser.
This plugin uses the PE parser to build the PE file representation (the aforementioned tree-like structure of Java objects) and then creates AST nodes and their aspects inside MPS.
The extended syntax brought by PE describes the layout of each element, e.g. it tells the plugin that one set of child nodes should be displayed horizontally, another set should be vertical with each child on a separate line and with some indentation, and so on.

A limitation of the PE4MPS approach is that it, from what we understand, does not implement full ANTLR syntax.
This means that every grammar might need a non-trivial adjustment before its usage.
One of our goals is to enable import of as many languages out-of-the-box as possible and adopting the full specification.

\paragraph{ANTLR{\_}MPS.}
\PV{Tenhle projekt byl z tech tri nejmene uzitecny a podle me i nejmene funkcni, tak mozna by se dal v ramci zkracovani vypustit. V dalsim projektu je na nej reference, tak treba ji vyhodit.}
ANTLR{\_}MPS~\cite{ANTLR2MPS} is another project that is dealing with a similar problem.
The author of this project is Fabien Campagne.
The ANTLR{\_}MPS project also uses ANTLRv4 grammar notation~\cite{ANTLR4} and tries to import grammars inside MPS.
However, it does not try to generate the projectional editor at all (i.e., it does not deal with this problem), probably because it is in an early stage of development.

The way this import process of ANTLR{\_}MPS works is quite different from what we have seen in the PE4MPS project, and it is quite complex/complicated to use.
It works as follows. We give only a brief high-level overview and omit low-level technical details.
The author created a whole new ANTLRv4 MPS language, which is an MPS port of the grammar notations' syntax.
To import a language, the user utilizes this MPS ANTLR language.
The textual grammar is imported automatically into MPS taking the form of the MPS's ported grammar language (so that the textual grammar is converted into MPS nodes, that means an AST node is created for each grammar rule).
No child-parent relationships in the structure are generated by the tool --- all must be created manually by the user.
There are no editor nor TextGen aspects created neither.

\paragraph{mps-metabnf.}
\PV{Pridal jsem popis projektu}
% TODO - citace na https://github.com/DSLFoundry/mps-metabnf
The mps-metabnf project comes from the DSLFoundry group and also takes on grammar importing.
Even though it is in a very early stage of development, it holds some interesting ideas.
Similarly to ANTLR{\_}MPS, authors of mps-metabnf decided to create an MPS language describing the grammar.
When grammar is being imported into MPS, it is just transformed into the terms of this MPS language.
User is then able to adjust this grammar.
After this, a second step of the import process is needed, that generates the final MPS language out of this adjusted grammar definition.

The key takeaways from the mps-metabnf project are that the intermediate grammar MPS representation, when finished, will offer some powerful means of adjustment.
Using the projectional editor, the user can easily specify information about the code layout, export the target MPS language and, in case further adjustments are needed, come back to the grammar definition, fix any problems and regenerate the language.
The MPS language of the grammar definition can be easily extended with needed features (indenting, line breaks...), which is superior to our approach, because they can be represented using many visual means.

This approach also has some downsides.
Because MPS doesn't offer any tools that would allow easy transformation from the MPS grammar \PP{To je konkretni gramatika zadefinovana pomoci MPS jazyka pro ANTLR} to the MPS language, the second step of the import process hasn't been implemented yet and will be probably very problematic.
Another downside might be the need for a double transformation implementation -- first one being the import of the text grammar to the MPS grammar definition.

From another point of view -- for each additional feature, that we decide to add to the grammar notation, we need to implement a corresponding transformation.
This subsequently leads to duplicating the projectional editor that is already present in MPS.
This means that there needs to be a line drawn between which features are added and which will be left for the user to add to the target language using MPS.



\section{Creating Language Definition in MPS}
\label{sect:LANGDEF}

The proposed INGRID method accepts grammars in the ANTLR v4 notation~\cite{ref:ANTLRBOOK} as input.
All widely used programming languages have their syntax defined using this notation~\footnote{https://github.com/antlr/grammars-v4}.
Unlike some of the related projects, we did not extend the notation with any custom features, and we also did not create an MPS language for the ANTLR notation.

The process of MPS language construction, as performed by INGRID, consists of four phases --- the first is parsing of the input grammar, which is followed by definition of the essential aspects of the language (Structure, Editor, and TextGen).
Each of the phases 2-4 is exclusively and fully responsible for one aspect.

INGRID is currently able to create (i) a full Structure aspect for each element (type of an AST node) of the given language, (ii) a very basic Editor, and (iii) a basic TextGen aspect.
Therefore, the resulting MPS language typically still has to be adjusted manually to improve its usability, and the human users must also define the remaining aspects not yet supported by INGRID.

Our approach differs from the related projects (Section~\ref{sect:RELATED}) especially in the level of automation. It also has much better support for the definition of Editor and TextGen.

\subsection{Running Example}

We will describe the INGRID method using the example of a simplified XML language that is defined in Figure~\ref{fig:SIMPLEXML}.
The \emph{SimpleXML} language is small but complex enough to be used for illustration of the main challenges and behavior of the proposed algorithms.

\begin{figure*}[ht]
\centering
\begin{framed}
\begin{alltt}
\textbf{grammar} \textit{SimpleXML};
\antlrparserrule{document}  : \antlrparserrule{prolog}? \antlrparserrule{comment}? \antlrparserrule{element} ;
\antlrparserrule{prolog}    : \antlrliteral{<?xml } \antlrparserrule{attrib}* \antlrliteral{?>} ;
\antlrparserrule{comment}   : \antlrliteral{<!--} \antlrlexerrule{TEXT} \antlrliteral{-->} ;
\antlrparserrule{element}   : \antlrliteral{<} \antlrlexerrule{Name} \antlrparserrule{attrib}* \antlrliteral{>} \antlrparserrule{content}* \antlrliteral{</} \antlrlexerrule{Name} \antlrliteral{>} | \antlrliteral{<} \antlrlexerrule{Name} \antlrparserrule{attrib}* \antlrliteral{/>} ;
\antlrparserrule{attrib}    : \antlrlexerrule{Name} \antlrliteral{="} \antlrlexerrule{TEXT} \antlrliteral{"} ;
\antlrparserrule{content}   : \antlrlexerrule{TEXT} | \antlrparserrule{element} | \antlrparserrule{comment} | \antlrlexerrule{CDATA} ;
\antlrlexerrule{Name}      : \antlrlexerrule{NameStartChar} \antlrlexerrule{NameChar}* ;
\textbf{fragment}
\antlrlexerrule{DIGIT}     : \antlrregex{[0-9]} ;
\textbf{fragment}
\antlrlexerrule{NameChar}  : \antlrlexerrule{NameStartChar} | \antlrliteral{-} | \antlrliteral{\_} | \antlrliteral{.} | \antlrlexerrule{DIGIT} ;
\textbf{fragment}
\antlrlexerrule{NameStartChar} : \antlrregex{[:a-zA-Z]} ;
\antlrlexerrule{TEXT}      : \antlrregex{~[<"]*} ;
\antlrlexerrule{CDATA}     : \antlrliteral{<![CDATA[} \antlrregex{.*?} \antlrliteral{]]>} ;
\end{alltt}
\end{framed}
\caption{Grammar of the SimpleXML language in the ANTLR v4 notation}
\label{fig:SIMPLEXML}
\end{figure*}

The grammar of SimpleXML contains elements of the kinds that are listed below.
Each color in Figure~\ref{fig:SIMPLEXML} corresponds to a different kind in a way that is indicated by the list item headers.
\begin{itemize}
	\item \textbf{ANTLR v4 keywords} are required by the notation.
	\item \antlrparserrule{\textbf{Parser rules}} describe the structure of a language.
		Alternatives on the right side of a rule are separated by the pipe character ($|$).
	\item \antlrlexerrule{\textbf{Lexer rules}} describe terminal symbols that the parser matches against the input string.
		A terminal symbol can be encoded as a string value or using a regular expression.
		The \textbf{fragment} keyword states that the lexer rule just brings more clarity into the notation but it is not visible in the parser output.
	\item \antlrliteralnoap{\textbf{String literals}}, always written inside of a pair of single quote marks, also represent terminal symbols, but by exact match to a string constant.
	\item \antlrregex{\textbf{Regular expressions}} describe string tokens to be matched against regular expressions with a special ANTLR v4 regex notation\footnote{https://github.com/antlr/antlr4/blob/master/doc/lexer-rules.md}.
\end{itemize}
Elements on the right side of a rule can be annotated with standard EBNF operators (\code{?}, \code{+}, \code{*}) that specify the allowed number of occurrences.

\subsection{Phase 1: Parsing Input Grammar}

The first phase in the process of MPS language construction is parsing of the input grammar that is defined using the ANTLR v4 notation.
An output of this phase is an AST representing the grammar.
Nodes of the AST correspond to rules and other grammar elements.

Nevertheless, the full AST, which comes out of the automatically generated parser, is quite complex and contains information not relevant for the INGRID algorithm.
In order to get a simple representation that is easy process by the later phases and keeps only information necessary for the construction of MPS languages, several steps of post-processing of the AST have to be performed in this phase.

A simplified tree of objects is derived from the AST in two passes over the grammar.
The result of the first pass is a tree where each node contains names of other referenced rules in the form of strings.
In the second pass, the whole tree is traversed and all the references are resolved to have a form of real pointers to other rule objects.

The representation of lexer rules (tokens) in the AST has to be simplified too.
Note that, in the ANTLR notation, lexer rules can be built from alternatives just like the parser rules --- see, for example, the lexer rule \antlrlexerrule{Name} from our SimpleXML language (Figure~\ref{fig:SIMPLEXML}).
For each lexer rule, the parser produces a tree that captures the structure of the rule.
However, the lexer rules are, in fact, just regular expressions used to recognize tokens in the input string.

Every tree that represents a lexer rule is flattened into the equivalent regular expression by the recursive algorithm specified in Figure~\ref{fig:ALGFLATTEN}.
A distinct sequence is created from the elements of each alternative, and then all those sequences are joined by the alternation operator \code{$|$} to form a regular expression.

\begin{figure}[ht]
\centering
\begin{framed}
\begin{alltt}
Flatten(\textit{R}):
  \textit{T} = empty list
  \textbf{for each} alternative \textit{A} \textbf{in} rule \textit{R}:
    \textit{R} = \antlrap\antlrap
    \textbf{for each} element \textit{E} \textbf{of} \textit{A}:
      \textbf{if} \textit{E} is not yet flattened \textbf{then}:
        Flatten(\textit{E})
      \textit{R}.append(\textit{E})
      \textit{R}.append(\textit{E}.operator)
    \textit{T}.add(\textit{R})
  build string S from elements of \textit{T}:
    \textit{S} = \(t\sb{1}\) | \(t\sb{2}\) | \(\ldots\) | \(t\sb{n}\)
  \textbf{return} \textit{S}
\end{alltt}
\end{framed}
\caption{Flattening algorithm}
\label{fig:ALGFLATTEN}
\end{figure}

The output of \texttt{Flatten(Name)}, i.e. application of the algorithm to the \antlrlexerrule{Name} rule, is this regular expression:

\begin{center}
  \texttt{[:a-zA-Z]([:a-zA-Z]|\textbackslash-|{\_}|\textbackslash.|[0-9])*}
\end{center}
It defines the syntactically valid identifiers of elements and attributes in SimpleXML.
Each identifier must begin with a letter, followed by any combination of letters, digits, underscore, dash, or a dot.

\subsection{Phase 2: Structure}

In the next phase, the complete structure of the MPS language is automatically generated from the AST that represents syntax of the input language.
Elements of the Structure aspect are derived from the AST nodes, and then linked appropriately.
Therefore, structure of the MPS language is usually similar to the original ANTLR grammar.

When designing the process for translating AST nodes (i.e., grammar rules) into the MPS language structure, we faced several challenges.
The main challenge related to the structure, as we already mentioned in the introduction, is that a grammar typically contains rules that do not directly correspond to programming language constructs.
Such rules exist at the intermediate layers of a syntax hierarchy.
Their main purpose is to enable easier understanding and maintenance of the grammar by humans.
However, presence of the intermediate layers significantly complicates usage of the given language in MPS, and the layers are actually not necessary for construction of an MPS language.
We show examples illustrating this problem, which we call a \emph{layer problem}, later in the subsection where we describe our solution.

As a part of the INGRID method, we have designed an approach to eliminate the unnecessary layers during construction of the Structure aspect --- we call it the \emph{shortcut} approach (Section~\ref{sect:SHORTCUT}).
However, before focusing on the intermediate layers and the shortcut approach, we describe the basic principles of translation from the AST into the language structure.

Each concept of the language (i.e., every AST node) is represented by an object that may have a parent, some children, and properties of any data type.
In addition, the object may implement any number of interfaces, and it may also contain references to objects representing other AST nodes.
The parent-child relationships between objects that make the Structure aspect are derived from rules of the grammar (i.e., from the structure of the AST).
For each rule, the object corresponding to the left-hand side is in the parent-child relationships with sets of objects that represent language elements referenced by the right-hand side.
If a rule has multiple alternatives, then a distinct object (MPS concept) has to be created in the structure for each alternative.
The name of a concept (object) in MPS is composed from (1) the name of the AST node, which is equivalent to the string encoding of the left-hand side of the corresponding rule, and (2) the number indicating the position of the respective alternative on the right-hand side of the rule.

\begin{figure}[ht]
\centering
\begin{alltt}
\small
\mpsstkeyword{concept} Element\_1 \mpsstkeyword{extends} BaseConcept
        \mpsstkeyword{implements} IContent, IElement

  \mpsstkeyword{instance can be root:} false
  \mpsstkeyword{alias:} \mpsstalias{< > </ >}
  \mpsstkeyword{short description:} \mpsstalias{Element}

  \mpsstkeyword{properties:}
  \mpsstproperty{Name\_1} : Name
  \mpsstproperty{Name\_2} : Name
  
  \mpsstkeyword{children:}
  \mpsstproperty{Attribute\_1} : Attribute[\mpsstcardinality{0..n}]
  \mpsstproperty{Content\_2}   : IContent[\mpsstcardinality{0..n}]
\end{alltt}
\caption{Structure aspect of the \code{Element{\_}1} concept}
\label{fig:ELEMENTSTRUCT}
\end{figure}

Consider the parser rule \antlrparserrule{element} from the grammar in Figure~\ref{fig:SIMPLEXML}.
Its first alternative represents the full XML element with content.
Figure~\ref{fig:ELEMENTSTRUCT} shows the Structure aspect for the MPS concept (object) named \code{Element{\_}1} that corresponds to the alternative.
The object contains two properties, one for each reference to the lexer rule \antlrlexerrule{Name}.
Values of these properties should be restricted using the regular expression that corresponds to the \antlrlexerrule{Name} rule, but this is left to the human user.
String literals, such as the keyword \textbf{for}, are omitted because they will be defined only in the Editor aspect for this concept.

References to other parser rules are captured by pointers to child objects.
The types of child objects, such as \code{IContent} in the \code{Element{\_}1} concept, are determined as follows.
Consider the \antlrparserrule{content} rule from the SimpleXML language.
We show just the relevant fragment of the grammar again in Figure~\ref{fig:CONTENTRULE}.
An object corresponding to any one of the four alternatives could be the actual value anywhere the \antlrparserrule{content} rule is referenced.

\begin{figure}[ht]
\centering
\begin{alltt}
\small
\antlrparserrule{content} : \antlrlexerrule{TEXT} | \antlrparserrule{element} | \antlrparserrule{comment} | \antlrlexerrule{CDATA} ;
\end{alltt}
\caption{Parser rule \antlrparserrule{content}}
\label{fig:CONTENTRULE}
\end{figure}

Our solution is to use interface concepts.
For each rule with more than one alternative on the right side, i.e. for each AST node with more than one child, first an interface concept is defined in the MPS language structure, and then one object that implements the given interface is created for each alternative.
The resulting fragment of the language structure looks like the one in Figure~\ref{fig:ICONTENTITF}.
It contains one interface \mpsinterface{IContent} that is implemented by four object concepts \mpsconcept{Content{\_}1}, $\ldots$, \mpsconcept{Content{\_}4}.

\begin{figure}[ht]
\centering
\begin{alltt}
\small
\mpsinterface{IContent} : \mpsconcept{Content{\_}1} | \mpsconcept{Content{\_}2} |
           \mpsconcept{Content{\_}3} | \mpsconcept{Content{\_}4}
\end{alltt}
\caption{MPS interface \code{IContent} and the types of implementing objects}
\label{fig:ICONTENTITF}
\end{figure}

Now we can illustrate the layer problem on the behavior of auto-completion in MPS.
Suppose that a user is creating a new document in the SimpleXML language, just inserted a fresh node of the type \mpsconcept{Element{\_1}} (see above), and would like to insert another nested XML element inside.
The auto-completion mechanism of MPS offers four options that are displayed in the left part of Figure~\ref{fig:LAYERPROBLEM}.
Each option represents one of the MPS concepts that implement the interface \mpsinterface{IContent}, and therefore also one alternative of the \antlrparserrule{content} rule.

\begin{figure*}[ht]
	\centering
	\includegraphics[scale=0.5]{./images/layer_problem.png}
	\caption{Layer problem in auto-completion}
	\label{fig:LAYERPROBLEM}
\end{figure*}

However, in order to correctly insert another nested element, a user has to perform two steps:
\begin{enumerate}
	\item Insert an object (node) of the type \mpsconcept{Content{\_}2} inside the \mpsconcept{Element{\_}1} node. The \mpsconcept{Content{\_}2} object has only a single child node of the interface type \mpsinterface{IElement}.
	\item Then, the user must trigger the auto-complete again and insert either an \mpsconcept{Element{\_}1} node or an \mpsconcept{Element{\_}2} node into the \mpsconcept{Content{\_}2} node. See the right part of Figure~\ref{fig:LAYERPROBLEM}.
\end{enumerate}
The main difficulty here is that, in the first step, the user has to (i) either correctly guess what option offered by the auto-complete menu to select, or (2) to remember the order of alternatives in the grammar rule.

Similarly, if the user would like to replace a nested \mpsconcept{Element{\_}1} node with, for example, an XML comment (represented by the \mpsconcept{Comment} node), then both intermediary layers have to be deleted before she gets back to the original selection among the concepts \mpsconcept{Content{\_}1}, $\ldots$, \mpsconcept{Content{\_}4}.

Intermediary layers have no visual appearance, and therefore it is very difficult for users to see what is actually happening and they may get confused very easily.
The layer problem is addressed by the shortcut approach, which we describe in the next subsection.

\subsubsection{The Shortcut Approach}
\label{sect:SHORTCUT}

The key idea of this approach is to skip all the intermediary layers (nodes) in the syntax tree and consider just the leaf nodes, e.g., to be directly offered to the user through the auto-completion menu.
Specifically, the \antlrparserrule{content} rule from the SimpleXML grammar, which is given in Figure~\ref{fig:CONTENTRULE} together with rules that determine the relevant fragment of the syntax hierarchy, expands ultimately into the leaf nodes that are highlighted using the bold font in Figure~\ref{fig:CONTENTEXPAND}.

\begin{figure*}[ht]
\centering
\begin{framed}
\begin{alltt}
  \antlrparserrule{content} : \antlrlexerrule{TEXT} | \antlrparserrule{element} | \antlrparserrule{comment} | \antlrlexerrule{CDATA} ;
  \antlrparserrule{element} : \antlrliteral{<} \antlrlexerrule{Name} \antlrparserrule{attrib}* \antlrliteral{>} \antlrparserrule{content}* \antlrliteral{</} \antlrlexerrule{Name} \antlrliteral{>} | \antlrliteral{<} \antlrlexerrule{Name} \antlrparserrule{attrib}* \antlrliteral{/>} ;
  \antlrparserrule{comment} : \antlrliteral{<!--} \antlrlexerrule{TEXT} \antlrliteral{-->} ;
\end{alltt}
\end{framed}
\caption{The parser rule \antlrparserrule{content} with other rules that determine the relevant fragment of the syntax hierarchy}
\label{fig:CONTENTRULE}
\end{figure*}

\begin{figure}[ht]
\begin{framed}
\textbf{Content{\_}1} (TEXT) \\
\ \ \ Content{\_}2 $\rightarrow$ \textbf{Element{\_}1} \\
\ \ \ Content{\_}2 $\rightarrow$ \textbf{Element{\_}2} \\
\ \ \ Content{\_}3 $\rightarrow$ \textbf{Comment} \\
\ \ \ \textbf{Content{\_}4} (CDATA)
\end{framed}
\caption{Leaf nodes of the parser tree fragment that has the \antlrparserrule{content} rule as its root}
\label{fig:CONTENTEXPAND}
\end{figure}

For the input that consists of a particular AST node $N$ and the grammar rule $R$ that expands $N$, the procedure implementing the shortcut approach systematically traverses the parser tree (AST) built in the earlier phases in order to identify each AST node that may appear at the end of some derivation chain starting by the rule $R$ from the node $N$.
Such nodes cannot expand further based on any rule in the grammar, and for that reason we call them \emph{end nodes}.

\begin{figure*}[ht]
\begin{framed}
\begin{alltt}
 1 FindAllPathsToEndNodes(\textit{R}):
 2   \textit{CurPath} = empty list of rules and nodes
 3   \textbf{return} FindPaths(\textit{R}, \textit{CurPath})
 4
 5 FindPaths(\textit{R}, \textit{CurPath}):
 6  \textit{Paths} = empty list
 7  \textbf{for each} alternative \textit{A} \textbf{in} rule \textit{R}:
 8    \textit{NewCurPath} = Clone(\textit{CurPath})
 9    \textbf{if} \textit{A} contains only a single element \textit{E}:
10      \textit{NewCurPath}.Add(\(N\sb{E}\)) \textbf{where} \(N\sb{E}\) is the node representing \textit{E}
11      \textit{P} = FindPaths(\(R\sb{E}\), \textit{NewCurPath}) \textbf{where} \(R\sb{E}\) is the rule that expands \textit{E}
12      \textit{Paths} = Merge(\textit{Paths}, \textit{P})
13    \textbf{else}:
14      \textit{NewCurPath}.Add(\textit{R})
15      \textit{NewCurPath}.Add(\(N\sb{A}\)) \textbf{where} \(N\sb{A}\) is the node representing \textit{A}
16      \textit{Paths}.Add(\textit{NewCurPath})
17  \textbf{return} \textit{Paths}
\end{alltt}
\end{framed}
\caption{Algorithm to find all paths to end nodes for a parser rule}
\label{fig:SHORTCUTALG}
\end{figure*}

Figure~\ref{fig:SHORTCUTALG} shows the algorithm that finds all paths to some end node from a given parser rule $R$.
The algorithm is based on recursive traversal of the parser tree.
At each level of recursion, it gathers all paths that lead from the current parser rule $R$ to some end node through its alternatives (line~7).
Two cases may occur:
\begin{itemize}
	\item When a particular alternative $A$ of the rule $R$ contains only a single element $E$, and the element is a reference to another parser rule (line 9), the alternative $A$ is an intermediary layer that can be transparently hidden from the user of the MPS language.
		  A run of the algorithm continues, at line 11, by recursively processing alternatives of the rule corresponding to $E$, which lies at the next level of the parser tree.
	\item Otherwise, an end node of a derivation chain was found and the recursion stops (lines 13-16).
\end{itemize}
By appending the node corresponding to the current alternative (lines 10 and 15) and the rule that leads to the alternative (line 14), the algorithm creates a full path that contains the target end node as the last element of the chain.

We use the name \emph{shortcut approach} for this algorithm, because the paths collected by the algorithm provide shortcuts from the given rule to end nodes, by the virtue of hiding all intermediary layers.
For example, the result of this algorithm for the \antlrparserrule{content} rule (Figure~\ref{fig:CONTENTRULE}) is the list of five items already shown in Figure~\ref{fig:CONTENTEXPAND}.

The primary use case for shortcuts is to generate options for the auto-completion menus, such that only the end nodes are offered.
Shortcuts have to be considered not only when nodes are inserted, but also when they are deleted.
In each case, the whole chain including possibly multiple intermediary AST nodes (up to the end node) must be added, respectively deleted.
When the user wants to delete some end node from the AST of a program or document written in the MPS language, the effect of an insertion must be fully reversed.

\subsection{Phase 3: Editor}

Having the complete structure of the new MPS language, the next phase is to define the visual representation of all concepts (AST nodes) in the projectional editor.
As we said in Section~\ref{sect:MPS}, MPS uses a cellular system that enables the language developer to arrange the children and properties of an AST node in a table-like manner.
Cells of different types are supported by MPS --- for storing property values, references to child nodes, and keywords (string literals), and also cells that influence layout (e.g., indentation).
The specific goal of this phase is to create the Editor aspect for each language concept (element), such that all attributes of the concept --- name, properties, children --- are projected using the respective cell types.

The main problem that we had to address is the absence of information about the code layout and whitespace in the ANTLR grammar of an input language.
Rules forming the grammar only tell what the syntax tree looks like and how the program code is decomposed into AST nodes.
Here we present a solution that is only partially automated for reasons explained below.

We experimented with several heuristics to derive a useful code layout, but all of them produced rather suboptimal results.
However, we observed that the most tedious and error-prone step in the manual definition of the Editor aspect is the creation of cells for all literals (keywords), properties, children, and other fields of a given concept that should appear in its visual representation.
This step can be very easily automated.

Our solution that we implemented in the current version of the INGRID method is to create all the cells and place them in a single row.
The resulting basic layout is illustrated on the example of the \mpsconcept{Element{\_}1} concept in the upper left corner of Figure~\ref{fig:EDITORADJUST}.
Further adjustments of the layout, such as indentation and line breaks, can be done very efficiently in the MPS IDE.
The bottom right part of Figure~\ref{fig:EDITORADJUST} shows a fully customized layout for the \mpsconcept{Element{\_}1} concept.

\begin{figure*}[ht]
	\centering
	\includegraphics[scale=0.55]{./images/editor_adjustment.png}
	\caption{Editor aspect of the \mpsconcept{Element{\_}1} concept}
	\label{fig:EDITORADJUST}
\end{figure*}

Based on our experience, it takes a very short time to manually adjust the layout into a form much better than any fully automated heuristic could achieve.
We expect that a typical user of the INGRID method will be able to use the projectional editor in MPS quite efficiently.
More details are provided in Section~\ref{sect:EVAL}, including our experience with several mainstream programming languages.

We have found that the combination of two steps, (1) automated placing of cells into a single row and (2) subsequent manual adjustment of the layout, is a very fast and efficient way of creating a nice code layout.
Nevertheless, we plan to work on a more automated approach to the definition of Editor aspects in the future, probably using some techniques of machine learning.
The key idea would be to derive the code layout from a set of valid input source files.
We believe that the user would have to adjust the editor manually a little bit even in this case, because the results of a machine learning-based approach would not be perfect.

\subsection{Phase 4: Text Generation}

The purpose of the last phase of the MPS language construction is to generate the TextGen aspect for each concept.
We use the \mpsconcept{Element{\_}1} concept again for illustration.
Figure~\ref{fig:TEXTGENBASIC} shows a very basic example of BaseLanguage code that can be used as its TextGen aspect.
The code appends all literals, properties, and children of \mpsconcept{Element{\_}1} to the output buffer, and does that in the same order as they appear in the corresponding grammar rule.

\begin{figure}[ht]
\begin{alltt}
\small
\mpstgkeyword{text gen component for concept} \mpstgtarget{Element{\_}1} \{
  \mpstgparam{(context, buffer, node)->void} \{
    \mpstgaction{append} \{\mpstgliteral{<}\};
    \mpstgaction{append} \$\{\mpstgparam{node}.\mpstgnodeprop{\textit{Name}}{\_}\textit{1}\};
    \mpstgaction{append} \{\ \};
    \mpstgaction{append} \$list\{\mpstgparam{node}.\mpstgnodeprop{\textit{Attribute}}{\_}\textit{1}\};
    \mpstgaction{append} \{\mpstgliteral{>}\};
    \mpstgaction{append} \$list\{\mpstgparam{node}.\mpstgnodeprop{\textit{Content}}{\_}\textit{2}\};
    \mpstgaction{append} \{\mpstgliteral{</}\};
    \mpstgaction{append} \$\{\mpstgparam{node}.\mpstgnodeprop{\textit{Name}}{\_}\textit{2}\};
    \mpstgaction{append} \{\mpstgliteral{>}\};
  \}
\}
\end{alltt}
\caption{Basic TextGen aspect for the \mpsconcept{Element{\_}1} concept}
\label{fig:TEXTGENBASIC}
\end{figure}

Like in the case of a projectional editor, the main challenge associated with TextGen is to produce valid code with a reasonable layout.
An implementation of the TextGen aspect must determine properly where to put line breaks, indentation, and other whitespace characters.
For example, in the case of the concept representing an XML element, there must be a space between the element's name and the first attribute, while it is not required between the opening bracket \antlrliteral{\textless} and the name (Figure~\ref{fig:TEXTGENBASIC}).

Our INGRID method targets mostly text-based languages, for whose concepts the visual representation in a projectional editor must be almost equivalent to their plain-text representation.
An obvious choice would therefore be to use the same approach for Editor and TextGen.
Nevertheless, since the Editor aspect has to be adjusted manually, we decided to use a different approach for TextGen.
We designed a procedure based on a simple fully automated heuristic that provides surprisingly good results.

The procedure creates the TextGen aspect for a given language concept (AST node) in two steps.
First, it generates a basic variant that inserts spaces in between every two tokens of the textual representation of the concept.
In the second step, which is the core of our heuristic, spaces are eliminated from places where they are not really needed.
The main criterion is whether the generated plain-text output can be safely processed by a parser of the original language (using the ANTLR grammar).
We discuss all the cases here:
\begin{itemize}
	\item When there is a non-alphabetical literal that is used as a token in the grammar, and that might get recognized by the parser without the need for whitespace separators around it, then we can omit the spaces. An example of such literal is '\textless' in SimpleXML.
	\item In the case of an arbitrary string, the whitespace may be omitted when the adjacent literal ends, respectively begins, with a non-alphabetical character.
		This applies especially to the values of properties defined in Structure aspects, such as the name of an XML element.
		Based on this heuristic, redundant spaces inside of quotes will be eliminated, as well as spaces next to semicolons and around brackets.
		On the other hand, it will distinguish language keywords (e.g., \code{function}, \code{var}, \code{in}) from other literals by preserving the space character in between them.
	\item Spaces can be safely omitted also when specific optional child nodes are not present, and in the case of empty lists of child nodes, so that whitespace does not accumulate.
\end{itemize}
A space must be alway inserted when two child nodes are next to each other.
Elements of a sequence of child nodes are separated with space or a line break.

\begin{figure}[ht]
\begin{alltt}
\small
\mpstgkeyword{text gen component for concept} \mpstgtarget{Element{\_}1} \{
  \mpstgparam{(context, buffer, node)->void} \{
    \mpstgaction{append} \{\mpstgliteral{<}\};
    \mpstgkeyword{if} (\mpstgparam{node}.\mpstgnodeprop{\textit{Name}}{\_}\textit{1}.\textbf{isNotEmpty}) \{
      \mpstgaction{append} \$\{\mpstgparam{node}.\mpstgnodeprop{\textit{Name}}{\_}\textit{1}\};
    \}
    \mpstgkeyword{if} (\mpstgparam{node}.\mpstgnodeprop{\textit{Attribute}}{\_}\textit{1}.\textbf{size} > 0) \{
      \mpstgaction{append} \{\ \};
      \mpstgaction{append} \$list\{\mpstgparam{node}.\mpstgnodeprop{\textit{Attribute}}{\_}\textit{1} with  \};
    \}
    \mpstgaction{append} \{\mpstgliteral{>}\};
    \mpstgkeyword{if} (\mpstgparam{node}.\mpstgnodeprop{\textit{Content}}{\_}\textit{2}.\textbf{size} > 0) \{
      \mpstgaction{append} \$list\{\mpstgparam{node}.\mpstgnodeprop{\textit{Content}}{\_}\textit{2} with  \};
    \}
    \mpstgaction{append} \{\mpstgliteral{</}\};
    \mpstgkeyword{if} (\mpstgparam{node}.\mpstgnodeprop{\textit{Name}}{\_}\textit{2}.\textbf{isNotEmpty}) \{
      \mpstgaction{append} \$\{\mpstgparam{node}.\mpstgnodeprop{\textit{Name}}{\_}\textit{2}\};
    \}
    \mpstgaction{append} \{\mpstgliteral{>}\};
  \}
\}
\end{alltt}
\caption{Full TextGen aspect for the \mpsconcept{Element{\_}1} concept}
\label{fig:TEXTGENFINAL}
\end{figure}

\begin{figure}[ht]
\begin{alltt}
\small
\mpstgkeyword{if} (\mpstgparam{node}.\mpstgnodeprop{\textit{Content}}{\_}\textit{2}.\textbf{size} > 0) \{
  \mpstgaction{append} \textcolor{Blue}{\textbf{\textbackslash{}n;}}
  \mpstgaction{indent buffer;}
  \mpstgkeyword{with indent} \{
    \mpstgaction{append} \$list\{\mpstgparam{node}.\mpstgnodeprop{\textit{Content}}{\_}\textit{2} with  \};
  \}
  \mpstgaction{append} \textcolor{Blue}{\textbf{\textbackslash{}n;}}
\} 
\end{alltt}
\caption{Fragment of the TextGen aspect of \mpsconcept{Element{\_}1} with adjusted indentation}
\label{fig:TEXTGENADJUSTED}
\end{figure}

Figure~\ref{fig:TEXTGENFINAL} contains the complete, automatically generated, TextGen aspect for the \mpsconcept{Element{\_}1} concept, which represents the full SimpleXML element.
Such code may be adjusted manually very easily in order to produce nicely indented XML documents.
We only need to wrap the \mpsconcept{Content{\_}2} child node with indentation and change the sequence separator to a new line character.
A fragment of the resulting adjusted aspect (code) is in Figure~\ref{fig:TEXTGENADJUSTED}.





% TODO continue from here



\subsection{Remarks about Grammars}

As we have already shown in previous sections, it is very easy to write a grammar in a way that will cause problems during import into MPS (or once it is imported).
Problems might occur during the creation of any aspect.
Here we will look at some general problems/issues that grammar import poses (might pose).
We will talk about these problems with respect to the challenges/obstacles we tried to overcome and which we described in previous sections (dedicated to structure, editor, and texgen, respectively).
We also show a few examples of additional possible complications.

\paragraph{Adjusting grammars.}
There are some cases, where altering the input grammar might yield far better MPS language.
We will show two examples how the usability of the resulting MPS language can be improved.
Let us look at the definition of an XML attribute in Figure~\ref{fig:xmlattribute}.

\begin{figure}[ht]
\centering
\begin{framed}
\begin{alltt}
	\antlrparserrule{attrib} : \antlrlexerrule{Name} \antlrliteral{=} \antlrlexerrule{STRING} ;
	\antlrlexerrule{STRING} : \antlrliteral{"} \antlrregex{~["]*} \antlrliteral{"}
	       | \antlrliteral{\textbackslash'} \antlrregex{~[']*} \antlrliteral{\textbackslash'} ;
\end{alltt}
\end{framed}
\caption{Definition of an XML attribute}
\label{fig:xmlattribute}
\end{figure}

The original XML grammar has quotes as a part of the value.
For the resulting MPS language, it would mean that there would be a placeholder for the attribute value that would expect us to input the leading and trailing quote together with the value too each time.
It would also be marked red (by the syntax checker) unless we enter both quotes inside the value since the regular expression checking for quotes will not match.
The user might be confused by this and will not be able to tell why his string value is incorrect.

In our SimpleXML language, we adjusted the grammar easily in the manner that we show in Figure~\ref{fig:xmladjustgrammar}.

\begin{figure}[ht]
\centering
\begin{framed}
\begin{alltt}
	\antlrparserrule{attrib} : \antlrlexerrule{Name} \antlrliteral{="} \antlrlexerrule{TEXT1} \antlrliteral{"}
	       | \antlrlexerrule{Name} \antlrliteral{=\textbackslash'} \antlrlexerrule{TEXT2} \antlrliteral{\textbackslash'} ;
	\antlrlexerrule{TEXT1} : \antlrregex{~["]*} ;
	\antlrlexerrule{TEXT2} : \antlrregex{~[']*} ;
\end{alltt}
\end{framed}
\caption{Adjusted grammar of SimpleXML}
\label{fig:xmladjustgrammar}
\end{figure}

We turned quotes into literals, thus ensuring that they will only appear in the projectional editor as fixed constant cells.
We will not have to encapsulate the value in them each time.
The user will only have to choose, which attribute version he wants to use (single or double quotes).

As the second example, we will use the ECMAScript\footnote{https://github.com/antlr/grammars-v4/blob/master/ecmascript/ECMAScript.g4} language otherwise known as JavaScript.
Every statement in JavaScript needs to be either followed by a semicolon, newline, file end or end of the block --- see the Figure~\ref{fig:javascriptstmt}.

\begin{figure*}[ht]
\centering
\begin{framed}
\begin{alltt}
	\antlrparserrule{eos} : \antlrlexerrule{SemiColon} | \antlrlexerrule{EOF} | {lineTerminatorAhead()}? 
	    | {{\_}input.LT(1).getType() == \antlrlexerrule{CloseBrace}}? ;
	\textcolor{gray}{// Example reference of the eos rule}
	\antlrparserrule{breakStmt} : \antlrlexerrule{Break} \antlrlexerrule{Identifier}? \antlrparserrule{eos} ;
\end{alltt}
\end{framed}
\caption{Statements in JavaScript}
\label{fig:javascriptstmt}
\end{figure*}

Because there are multiple options, our import plugin would create a placeholder at the end of every statement.
Every language element representing a statement would have one child of the \mpsinterface{IEos} interface type.
This placeholder needs to be manually filled in for each statement.

Since the projectional editor has much bigger power over the form of the code, we (a user) might want to have each statement on a separate line.
Since we can differentiate between statements on the AST level, we do not need an explicit separator between them.
This means that we might want to simplify the language and leave the semicolon out, or leave it just as a constant fixed part of the projectional editor, but not as something the user must explicitly fill in.
Then we can just put each statement on a separate line as it is usual for JavaScript code, but we do not need the semicolon anymore.
This small adjustment is very quick when done inside the grammar.
We just change the \antlrparserrule{eos} rule to the form shown in Figure~\ref{fig:eosrule}.

\begin{figure}[ht]
\centering
\begin{framed}
\begin{alltt}
	\antlrparserrule{eos} : \antlrliteral{;} ;
\end{alltt}
\end{framed}
\caption{eos parser rule}
\label{fig:eosrule}
\end{figure}

This way, we can change the grammar in a way that it will not describe the same ECMAScript language as before, but will definitely make our MPS language more usable.
TextGen aspect may be defined in such a way that puts the semicolon back (after each statement) in the generated plain-text representation.
This would be very important especially when JavaScript is available as another base language (which is actually a plan (future work) of the JetBrains company).

Above, we have shown that adjusting the input grammar might be a very fast mean of tweaking/improving the usability of generated MPS language, and sometimes it may be the only proper solution in some complex situations.
In general, the question we can ask ourselves is whether one aims for a full precise/faithful port of the input language or for an MPS alternative.

\todo{Nahradit slovo "break" v nasledujicich odstavcich (v kontextu "break the parser") za vhodnejsi slovo ktere pusobi vic odborne.}

\paragraph{Breaking grammars and parsers.}
However, we have found that there is one big problem with grammar adjustment that we would like to point out.
The problem is that it is very easy to change the input ANTLR grammar in a way that breaks the ANTLR parser generated out of it.
By breaking we mean that it stops parsing the original language.
We illustrate the problem on our SimpleXML language.

As stated before, when creating the SimpleXML grammar, we have started off with the original XML grammar\footnote{https://github.com/antlr/grammars-v4/tree/master/xml} and did some adjustments to it.
We ended up with a grammar that can be processed by our method/plugin and creates a nice usable MPS language.
Nevertheless, we noticed that even though the imported language behaves well enough and mimics the XML language quite nicely, the ANTLR parser generated out of this grammar no longer parses XML successfully.
Some changes we have made, such as the attribute adjustment, broke the grammar down.
More precisely, our changes improved the MPS language while breaking down the parser, and we have not even noticed it because, from the perspective of MPS, the imported language still corresponds to XML.

What we are trying to say (show here) is that it is very easy to perform a grammar adjustment that will, at first, seem harmless and valid inside MPS, but will break the parser.
The problem is that the user performing this change probably will not be aware of breaking it.

The main (underlying) cause of this problem is the way the ANTLR parser is implemented and the quite different purpose we are using the grammar for in our project/tool/method.
There are many ways how various parsers deal with, for example, token matching.
For instance, ANTLR introduces so called \emph{greedy} and \emph{non-greedy} operators.
The greedy way, in which ANTLR matches input on defined tokens and prioritizes their selection, makes some rules very dangerous.
When a rule matches a wide range of input, it might happen during parsing that it is prioritized over other rules and swallows a lot more input than the author of it intended.
Usually, these dangerous rules are bound to some parser context, which makes them behave well.

However, for practical reasons, it is very important that grammar is not broken by our custom adjustments (and that the ANTLR parser of the language works properly), because we might need to automatically generate parsers of the adjusted language.
Imagine this very real scenario (that will be supported in the future):
\begin{enumerate}
	\item We have imported the language inside MPS.
	\item MPS now knows the structure of the language, and we can code in MPS using this language.
	\item We would expect MPS to be able to load an existing source code, written in this language, from a text file and import it inside MPS.
	\item We would like to use MPS to safely edit this code, using all the features of MPS.
	\item We would also like to export the code in a plain-text form again and save it back to the source file.
\end{enumerate}
In order to make this happen, we must be able to generate a correctly-working ANTLR parser out of the adjusted source grammar that we used to (that helped us) import the language.
We must also be able to match nodes of the AST coming out of the ANTLR parser to elements of the MPS language, and we must build the MPS AST out of the ANTLR AST.

\section{Evaluation}
\label{sect:EVAL}

We implemented the proposed INGRID method as an MPS plugin that allows users to generate languages from ANTLR v4 grammars.
While most of the plugin is written in Java, small fragments of BaseLanguage code were needed to call the MPS API, e.g. in order to programmatically create new language concepts and their aspects.
The plugin uses the ANTLR library~\cite{ref:ANTLR} for parsing of ANTLR v4 grammar files, and several MPS libraries that implement the MPS API.
Our complete implementation is available at \url{https://github.com/premun/ingrid}.

For the purpose of evaluation, we applied INGRID to several well-established and widely used languages, including JSON, JavaScript, and C\#.
We used the version of JavaScript that is standardized as ECMAScript 5.1~\cite{ref:ECMASCRIPT51} --- the specification was finalized in June 2011, and it is currently the most frequently adopted version.
MPS projects that contain definitions of all three languages are also released in the public repository (\url{https://github.com/premun/ingrid}).
In the rest of this section, we discuss our experience with application of INGRID to these languages, and then we highlight few general observations.

However, first we must emphasize that MPS languages automatically produced by the INGRID method, as defined in this paper, are not ready-to-use full-fledged MPS counterparts of the original input languages.
The structure of such a generated MPS language fully corresponds to the respective ANTLR grammar, but its other aspects (e.g., the Editor) would have to be manually tweaked (or defined from scratch) by the end user.
INGRID does not yet support advanced features of MPS, such as the aspect responsible for type checking and inference.
For each of the three languages (JSON, JavaScript, and C\#), we provide (1) its MPS definition exactly in the form created by INGRID and (2) the adjusted ANTLR v4 grammar used as input for INGRID, both in the repository that contains also our implementation.

\paragraph{JSON.}
The least amount of manual adjusting after the import into MPS was needed in case of the JSON language~\cite{ref:JSON}, because it is the simplest language from all that we used for our experiments.
Specifically, the first author spent less than 20 minutes in order to get a language that is ready to use.

\paragraph{JavaScript.}
In the case of JavaScript, which is an example of a widely-used complex general purpose language, automated generating of the language definition in MPS from the ANTLR grammar\footnote{https://github.com/antlr/grammars-v4/blob/master/ecmascript/ECMAScript.g4} and subsequent manual adjusting was done in less than one hour.

We are aware of other projects that aim to create a manual port of JavaScript into MPS.
For example, there is ECMAScript4MPS~\cite{ref:ECMA4MPS} developed by the author of the PE4MPS project~\cite{ref:PE4MPS} that we described in Section~\ref{sect:RELATED}.

The main advantage of INGRID over ECMAScript4MPS is that, despite its current limitations, INGRID fully automatically produces the definition of JavaScript in MPS that needs just minor adjustments to be really useful.
INGRID achieved a very good result especially in the case of language structure, concept aliases, and support for auto-completion.
The Structure aspect generated by INGRID is very similar to that of the ECMAScript4MPS project, which was created manually in a large number of hours.






% TODO continue from here



\paragraph{C\#}.
Pred generovanim MPS jazyka pro C\# bylo nutne upravit gramatiku (coz zabralo asi hodinu prace) aby INGRID vyprodukoval pomerne rozumny vysledek.
The definition of C\# in MPS is quite large and complex.
Obsahuje asi vic nez 800 konceptu (zkontrolovat).
Import of the adjusted C\# language funguje docela hezky (ale je hodne velky a dlouho se kompiluje).
Ale na skutecne dobre pouzitelny jazyk C\# uvnitr MPS by to chtelo jeste vic manualnich uprav gramatiky a take urcitych aspektu vysledneho MPS jazyka (editor, textgen, apod).

\paragraph{Other languages.}
We also experimented with import of other languages, such as Python and Ruby.
Import jazyku Python a Ruby moc nefunguje protoze gramatika je napsana stylem ktery neni vhodny pro INGRID.
INGRID cannot handle cyclic rules (which could be found in Python grammar, for example).
Cyklicka pravidla to tedy neumi, jen na ne upozorni (a rekne, ktere pravidlo je obsahuje).
Museli bychom gramatiku rucne upravit (bude to ale nase future work).

\paragraph{General observations.}
Most of the languages (that we tried to define in MPS, use/import into MPS) are quite complex when it comes to their structure and the overall syntax variety, so defining them manually would be a very tedious, error-prone and time-consuming process.
\todo{Discuss where INGRID helps the most based on the experience.}
(Our experience/experiments show that) Automated generation of the Structure aspect is (the part) where INGRID spares the user from many hours of tedious and sometimes quite challenging work.
On the other hand, we found out that manual tweaking/adjusting of the Editor and Textgen aspects (adding line breaks and indentation) is very effective, and takes only a relatively little/short time (much less than we expected).
The imported language can be, in our opinion, adjusted very fast into a more readable form.
The generated MPS language can be customized very fast into a very usable form.
In fact/total, the adjusted version of each language was created from the original import by the first author, who spent no more than between 20 to 60 minutes of working on each of them.
The first author spent only a couple of minutes of (manually) adjusting the languages after the automated import.
Our experience with all the languages confirms our hypothesis (that code layout in the Editor aspect can be adjusted very fast, from the basic default single-row layout), and justifies some of the decisions we have described in Section X (editor aspect) and Section Y (textgen), proving that our approach is quite practical.
It takes a very short time to reach optimal results (because MPS allows doing this very efficiently), as we show on the example of importing the JavaScript language and manually adjusting all editors that needed it, all that in a less than hour time.
\todo{Mozna rict:} However, during our work on this project, we also tried to design some/various automated heuristics, but all that we came up with so far give rather suboptimal/bad results especially when compared to results produced by a human user.


\section{Conclusion and Future Work}

The proposed INGRID method is only partially-automated, but nevertheless it reduces the amount of necessary manual work/effort (that the user has to perform) to a great degree.

\todo{Highlight again:} The challenges and problems discussed in this paper, and our solution, are not specific to MPS --- they are common for everyone tackling a similar problem (in the field of DSL and language workbenches).

\todo{Write something.}

There is lot of further follow-up work to be done. Many open problems remain.

Besides using some kind of automated (machine) learning to generate the Editor aspect, we would also like to do XYZ.
Generating nicely designed and user-friendly editors, maybe using artificial intelligence (machine learning) that would evaluate (analyze) existing code in plain text and based on that create the editors.
Moreover, once we would have some information about the layout for generating better projectional editor, we could leverage the same information and use it to improve TextGen.

Follow-up work might look more into the problem of code layout detection, possibly exploring suggested approaches such as the learning approach described in Section XY \todo{Tohle se snad podari aspon castecne implementovat}.
This could lead to better results in editor and TextGen aspect generation which could definitely use some improvement.
Another future endeavors will definitely touch the subject of generating parsers that would enable the user to import existing source code into MPS.



\acks
% TODO
This research was supported by JetBrains. \todo{Pridat jeste nejaky grant a kompletne preformulovat.}

\bibliographystyle{abbrvnat}

\begin{thebibliography}{}

\bibitem{ref:BISONFLEX} J. Levine. Flex \& Bison: Text Processing Tools. O'Reilly Media, 2009.

\bibitem{ref:ANTLRBOOK} T. Parr. The Definitive ANTLR 4 Reference. Pragmatic Bookshelf, 2013.

\bibitem{ref:MBEDDR} M. Voelter, D. Ratiu, B. Schaetz, and B. Kolb. mbeddr: an Extensible C-based Programming Language and IDE for Embedded Systems. SPLASH 2012, ACM.

\bibitem{ref:ANTLR} ANTLR parser generator, \url{http://www.antlr.org}

\bibitem{ref:ANTLR2MPS} F. Campagne. The ANTRL{\_}MPS project, \url{https://github.com/CampagneLaboratory/ANTLR_MPS}

\bibitem{ref:MPSMETABNF} DSLFoundry. The mps-metabnf project, \url{https://github.com/DSLFoundry/mps-metabnf}

\bibitem{ref:MPS} JetBrains MPS, \url{https://www.jetbrains.com/mps/}

\bibitem{ref:PE4MPS} M. Lombardo. The PE4MPS project, \url{https://github.com/mar9000/pe4mps}

\bibitem{ref:PE} M. Lombardo. PE grammars, \url{https://github.com/mar9000/pe}

\bibitem{ref:BASELANG} MPS BaseLanguage, \url{https://confluence.jetbrains.com/display/MPSD34/Base+Language}



% TODO opravit formatovani nasledujicich polozek a vhodne zaradit


\bibitem{ref:ANTLR4} ANTLRv4, \url{https://github.com/antlr/antlr4}

\bibitem{ref:ANTLR4reference} T. Parr. The Definitive ANTLR 4 Reference. The Pragmatic Bookshelf, 2013

\bibitem{ref:ANTLR4grammars} ANTLRv4 grammar repository, \url{https://github.com/antlr/grammars-v4}

\bibitem{ref:ECMAScript4MPS} ECMAScript4MPS project, \url{https://github.com/mar9000/ecmascript4mps}

\bibitem{ref:javascript} JavaScript ANTLRv4 grammar, \url{https://github.com/antlr/grammars-v4/blob/master/ecmascript/ECMAScript.g4}

\end{thebibliography}

\end{document}

