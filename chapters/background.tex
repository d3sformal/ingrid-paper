\section{Background and Related Work}
\label{sect:BACKGRELWORK}

In this section, first we introduce basic features of the MPS platform, and then we discuss other existing projects that aim to add support for general purpose languages into MPS.

\subsection{JetBrains MPS}
\label{sect:MPS}

JetBrains MPS is a complete language workbench --- an integrated development environment that allows developers to create their own languages and use them to write code.
The code can then be transformed into a target language, typically a GPL, such as Java or C, and eventually compiled into executable programs.

As we already indicated, MPS differs from typical IDEs in one important aspect --- the projectional editor.
The developer does not work with the textual representation of the source code directly, but rather with its AST that is the model of the code.
Basically, when using the MPS projectional editor, programs are created by assembling the tree (AST) out of predefined building blocks of selected languages.
The definition of a language in MPS dictates, where in the AST can certain elements be placed and how they can be nested inside other elements.
On the other hand, in traditional IDEs it is the parser that constructs an AST of a program using the language's elements.

The building blocks of MPS models are called nodes.
Code of any program in MPS is built from nodes, which represent instances of concepts defined in one of the languages that the program declares to be using.
In the context of MPS, a concept is, in fact, a building block of a language definition.
We use the terms \emph{MPS concept} and \emph{AST node} when needed to avoid confusion.

One of the key advantages of projectional editing stems from the separation of abstract and concrete syntax.
While AST provides a complete and precise representation of the code, the way it is displayed on the screen and the way the user interacts with it are unconstrained.
The editor can take any visual form and shape.
The language author can define multiple alternative visualizations and let the developer choose one that fits best the task at hand --- for example, editing, debugging, reviewing, resolving merge conflicts, etc.
In particular, the visual representations are not bound to be just textual at all.
For example, blocks corresponding to branches of an \code{if-then-else} statement can be aligned next to each other and displayed with different background colors.

The definition of an MPS concept (language element) consists of several aspects, where each aspect codifies a different part of the AST nodes' behavior.
The essential aspects are the following: \emph{Structure} (abstract syntax), \emph{Editor} (concrete syntax) and \emph{TextGen}.
Editor defines the concrete syntax (i.e., how the code is visualized and edited) and TextGen specifies how AST nodes are transformed into textual representation.
If, instead of generating text directly, programs in the language are supposed to be transformed into another language that is available in MPS, the Generator aspect must be used to specify the model-to-model transformation rules.
Since only the Structure, Editor and TextGen aspects are relevant for the contribution of this paper, we describe these three below in more detail, and neglect other aspects such as type system and data-flow.

Languages are built from the concepts using techniques known from object-oriented programming --- containment, inheritance, interfaces, and so on.
Therefore, a definition of a whole language in MPS typically has an object-oriented and hierarchical nature.

\paragraph{Structure.}
The fundamental aspect of any MPS language is Structure.
It must be created first for each intended element (concept) of the language.
Structure specifies core attributes of an MPS concept such as the name, inheritance relationships, possible child concepts (including their types and cardinalities), the implemented interfaces, and references to other AST nodes.
% other properties (fields of any type that can hold values), etc.
Figure~\ref{fig:if_statement_structure} shows definition of the Structure aspect for the \code{if-else} statement.

The definition of Structure restricts the type of AST nodes that can appear at a particular place in the tree.
For example, one can restrict the condition in the \code{if-then-else} statement to be a boolean expression, and the \code{then}-block to be a list of statements.

\begin{figure}[ht]
\centering
\begin{alltt}
\small
\mpsstkeyword{concept} IfStatement \mpsstkeyword{extends} Statement
        \mpsstkeyword{implements} IContainsStatementList
                   IDontSubstituteByDefault
                   IConditional

  \mpsstkeyword{instance can be root:} false
  \mpsstkeyword{alias:} \mpsstalias{if}
  \mpsstkeyword{short description:} \mpsstplaceholder{<no short description>}

  \mpsstkeyword{properties:}
  \mpsstproperty{forceOneLine}   : boolean
  \mpsstproperty{forceMultiLine} : boolean
  
  \mpsstkeyword{children:}
  \mpsstproperty{condition}        : Expression[\mpsstcardinality{1}]
  \mpsstproperty{ifFalseStatement} : Statement[\mpsstcardinality{0..1}]
  \mpsstproperty{ifTrue}           : StatementList[\mpsstcardinality{1}]
  \mpsstproperty{elsifClauses}     : ElsifClause[\mpsstcardinality{0..n}]
  
  \mpsstkeyword{references:}
  \mpsstplaceholder{<< ... >>}
\end{alltt}
\caption{Structure aspect of the \code{if-else} statement}
\label{fig:if_statement_structure}
\end{figure}

\paragraph{Editor.}
The Editor aspect is where the language designer specifies what the projectional editor representation of a code fragment (an AST) looks like on the screen and how the user interacts with the code.
JetBrains have developed a cellular system that enables placing properties and children of a node (concept) into different cells.
The author usually incorporates all of the node's children, references, and properties inside the representation, so that the future users of the language can insert all values that the node expects.
Additionally, cells of the editor can be styled using a language similar to CSS.
The supported visual characteristics include text color and indentation.

Figure~\ref{fig:if_editor_definition} shows an example of what the definition of the Editor aspect for the \code{if-else} statement might look like.
While here we indicate positions of cell borders on each line by spaces (just for the purpose of illustration), MPS GUI actually uses a graphically much more appealing way of displaying the Editor aspects, which involves vertical and horizontal lines of different colors and also background colors other than white for some cells.
The symbols \verb|[-| and \verb|-]| represent layout information (cells), which define rules for mutual positioning of the contained cells (vertical, horizontal, indentation).

\begin{figure}[ht]
\centering
\begin{alltt}
\small
\mpsedannotation{<default>} \mpsedkeyword{editor for concept} \mpsedtarget{IfStatement}
  \mpsedkeyword{node cell layout:}
    [-
      \mpsedkeyword{if} \textbf{(} \% conditions \% \textbf{)} [-
      \textbf{\{}
      [- \% ifTrue % -]
      \textbf{\}}
    -]
    ?[- \mpsedkeyword{else} \% ifFalseStatement \% -]
    -]
\end{alltt}
\caption{Editor aspect for the \code{if-else} statement}
\label{fig:if_editor_definition}
\end{figure}

\paragraph{BaseLanguage.}
Another important feature of MPS that we need to describe in more detail is \emph{BaseLanguage}~\cite{ref:BaseLanguage}.
It is a clone of Java implemented using the MPS constructs.
BaseLanguage was developed in the early days of MPS in order to implement MPS itself and also to support the basic set of language-definition DSLs,
Although BaseLanguage is syntactically almost identical to Java, it is edited in a projectional editor, just like all the languages in MPS.
The language-definition DSLs, used to define custom languages by their authors, are generated into the BaseLanguage.
Similarly, all the custom DSLs that are meant to be generated into Java choose BaseLanguage as their generation target.
The conversion to textual Java is handled by BaseLanguage without any further manual effort, since BaseLanguage has a TextGen aspect defined, which translates BaseLanguage code into textual Java sources.

\paragraph{TextGen.}
The TextGen aspect specifies how a given AST node will be translated into plain text representation.
It is typically needed only for the bottom-line base languages.
DSLs, on the other hand, need to define rules for model-to-model conversions (Generators), since these are rarely converted to text directly.
After TextGen has generated textual sources from an AST, a compiler for the particular GPL is invoked to compile the textual sources into binary code.
The TextGen definition follows a very straightforward pattern --- each node outputs its textual representation into a buffer, while calling TextGen of its children nodes at the right moments.
MPS calls the corresponding method on the root AST nodes of the given program.

\begin{figure}[ht]
\centering
\begin{alltt}
\small
\mpstgkeyword{text gen component for concept} \mpstgtarget{IfStatement} \{
  \mpstgparam{(context, buffer, node)->void} \{
    \mpstgaction{append} \textcolor{Blue}{\textbf{\textbackslash{}n;}}
    \mpstgaction{indent buffer;}
    \mpstgaction{append} \{\mpstgliteral{if (}\} \$\{\mpstgparam{node}.\mpstgnodeprop{condition}\} \{\mpstgliteral{) \{}\};
    \mpstgkeyword{with indent} \{
      \mpstgaction{append} \$\{\mpstgparam{node}.\mpstgnodeprop{ifTrue}\};
    \}
    \mpstgaction{append} \textcolor{Blue}{\textbf{\textbackslash{}n}} \{\mpstgliteral{\}}\} \$list\{\mpstgparam{node}.\mpstgnodeprop{elsifClauses}\};
    \mpstgkeyword{if} (\mpstgparam{node}.\mpstgnodeprop{ifFalseStatement}.\textbf{isNotNull}) \{
      \mpstgaction{append} \{ \mpstgliteral{else}\} \$\{\mpstgparam{node}.\mpstgnodeprop{isFalseStatement}\};
    \}
  \}
\}
\end{alltt}
\caption{TextGen aspect definition for the \code{if-else} statement}
\label{fig:if_statement_textgen}
\end{figure}

TextGen aspect for each AST node (concept of the language) has to be defined using the BaseLanguage.
Again, we include an example for the \code{if-else} statement (Figure~\ref{fig:if_statement_textgen}).

\subsection{Related Projects}

The BaseLanguage~\cite{ref:BaseLanguage} is an almost full port of the Java language, extended with MPS-specific features.
It was created manually by JetBrains, and it is still undergoing changes as Java itself is evolving.

The C language has also been manually tailored for MPS within the mbeddr project~\cite{ref:MBEDDR}.

Besides that, we are aware of three other projects that provide certain support for GPLs in the context of MPS --- namely PE4MPS, ANTLR{\_}MPS, and mps-metabnf.
We describe their main features and limitations in this section.

\paragraph{PE4MPS.}
PE4MPS~\cite{ref:PE4MPS} is a project that addresses the lack of information about code layout in grammars by using so called PE grammars~\cite{ref:PE}, which is a new grammar notation proposed by the same author.
The abbreviation PE stands for projectional editing.

PE grammars extend the syntax of the ANTLR v4 notation by custom constructs that provide information about the possible layout of AST nodes.
The current version, as of November 2016, supports just horizontal lists, vertical lists, and some indentation rules.
However, even these few features make the already complex syntax of ANTLR v4 much more complicated.

An MPS language is created by the PE4MPS tool in two steps.
First, the PE parser is used to load an input PE file and build its intermediate representation, which has the form of a tree-like structure consisting of Java objects.
Then, all the concepts (AST nodes) and their aspects are created inside MPS.
The information about the code layout, extracted from the PE file, is used when generating the projectional editor for every AST node.

Another limitation of PE4MPS is that it does not implement the full ANTLR syntax.
Every grammar might require non-trivial adjustments before it can be processed using the PE4MPS tool.
To compare, the goals behind our INGRID approach are (1) to enable import of as many languages as possible out-of-the-box with maximal automation and (2) to adopt the full specification of every imported language.





% TODO continue from here



\paragraph{ANTLR{\_}MPS.}
\PV{Tenhle projekt byl z tech tri nejmene uzitecny a podle me i nejmene funkcni, tak mozna by se dal v ramci zkracovani vypustit. V dalsim projektu je na nej reference, tak treba ji vyhodit.}
ANTLR{\_}MPS~\cite{ANTLR2MPS} is another project that is dealing with a similar problem.
The author of this project is Fabien Campagne.
The ANTLR{\_}MPS project also uses ANTLRv4 grammar notation~\cite{ANTLR4} and tries to import grammars inside MPS.
However, it does not try to generate the projectional editor at all (i.e., it does not deal with this problem), probably because it is in an early stage of development.

The way this import process of ANTLR{\_}MPS works is quite different from what we have seen in the PE4MPS project, and it is quite complex/complicated to use.
It works as follows. We give only a brief high-level overview and omit low-level technical details.
The author created a whole new ANTLRv4 MPS language, which is an MPS port of the grammar notations' syntax.
To import a language, the user utilizes this MPS ANTLR language.
The textual grammar is imported automatically into MPS taking the form of the MPS's ported grammar language (so that the textual grammar is converted into MPS nodes, that means an AST node is created for each grammar rule).
No child-parent relationships in the structure are generated by the tool --- all must be created manually by the user.
There are no editor nor TextGen aspects created neither.

\paragraph{mps-metabnf.}
\PV{Pridal jsem popis projektu}
% TODO - citace na https://github.com/DSLFoundry/mps-metabnf
The mps-metabnf project comes from the DSLFoundry group and also takes on grammar importing.
Even though it is in a very early stage of development, it holds some interesting ideas.
Similarly to ANTLR{\_}MPS, authors of mps-metabnf decided to create an MPS language describing the grammar.
When grammar is being imported into MPS, it is just transformed into the terms of this MPS language.
User is then able to adjust this grammar.
After this, a second step of the import process is needed, that generates the final MPS language out of this adjusted grammar definition.

The key takeaways from the mps-metabnf project are that the intermediate grammar MPS representation, when finished, will offer some powerful means of adjustment.
Using the projectional editor, the user can easily specify information about the code layout, export the target MPS language and, in case further adjustments are needed, come back to the grammar definition, fix any problems and regenerate the language.
The MPS language of the grammar definition can be easily extended with needed features (indenting, line breaks...), which is superior to our approach, because they can be represented using many visual means.

This approach also has some downsides.
Because MPS doesn't offer any tools that would allow easy transformation from the MPS grammar \PP{To je konkretni gramatika zadefinovana pomoci MPS jazyka pro ANTLR} to the MPS language, the second step of the import process hasn't been implemented yet and will be probably very problematic.
Another downside might be the need for a double transformation implementation -- first one being the import of the text grammar to the MPS grammar definition.

From another point of view -- for each additional feature, that we decide to add to the grammar notation, we need to implement a corresponding transformation.
This subsequently leads to duplicating the projectional editor that is already present in MPS.
This means that there needs to be a line drawn between which features are added and which will be left for the user to add to the target language using MPS.


