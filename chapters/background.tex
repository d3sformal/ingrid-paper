\paragraph{PE4MPS.}
PE4MPS~\cite{ref:PE4MPS} is a project that addresses the lack of information about code layout in grammars by using so called PE grammars~\cite{ref:PE}, which is a new grammar notation proposed by the same author.
The abbreviation PE stands for projectional editing.

PE grammars extend the syntax of the ANTLR v4 notation by custom constructs that provide information about the possible layout of AST nodes.
The current version, as of November 2016, supports just horizontal lists, vertical lists, and some indentation rules.
However, even these few features make the already complex syntax of ANTLR v4 much more complicated.

\PV{Nasledujici dve vety byly vzaty z popisu INGRID, tak funguje INGRID, ne PE4MPS.}
An MPS language is created by the PE4MPS tool in two steps.
First, the PE parser is used to load an input PE file and build its intermediate representation, which has the form of a tree-like structure consisting of Java objects.
Then, all the concepts (AST nodes) and their aspects are created inside MPS.
The information about the code layout, extracted from the PE file, is used when generating the projectional editor for every AST node.

The main disadvantage of the PE4MPS approach is that it only shifts the tedious manual work from creating projectional editors in the MPS IDE to writing PE grammars in plain text.
\PV{Navrhuji pridat (aby to bylo vice jasne?): User is forced to input the layout information by extending the grammar manually.}
Usage of a standard text editor is inferior (and much more error-prone) with respect to MPS, which has been designed for the purpose of specifying the code layout and provides lot of support to developers.
Another limitation of PE4MPS is that it does not implement the full ANTLR syntax.
Every grammar might require non-trivial adjustments before it can be processed using the PE4MPS tool.
To compare, the goals behind our INGRID approach are (1) to enable import of as many languages as possible out-of-the-box with maximal automation and (2) to adopt the full specification of every imported language.

\paragraph{ANTLR{\_}MPS.}
The ANTLR{\_}MPS project~\cite{ref:ANTLR2MPS} also uses the ANTLR v4 grammar notation, but otherwise works quite differently from PE4MPS.
We provide only a brief mention here because this project is in an early stage of development.
Its author created an ANTLRv4 MPS language, which captures the syntax of ANTLRv4 notation using MPS concepts.
Given some language as input, the textual grammar of the language is imported automatically into MPS, taking the form of the ANTLRv4 MPS language.
The next step would be to create a new MPS language from the grammar definition in MPS, but it is not implemented yet.

However, the project has also other important limitations.
The tool does not generate any parent-child relationships in the Structure aspect, and it completely neglects the Editor and TextGen aspects.
\PV{Dalsi veta mozna zbytecna?}
In particular, it does not support creating of the projectional editor at all.

\PV{Navrhuji pridat neco jako (ale mozna uz o tom mluvime jinde, jen si ted nejsem jisty):
The major take away from this project is the idea of an intermediate grammar representation inside of MPS which we have also considered.
The advantage of this could be the possibility of extending the ANTLRv4 notation using some graphic tools that MPS has to offer. 
Using the projectional editor, a user could add the layout information in an effective way.
The disadvantage is the dual implementation that is needed.
First to transform the ANTLRv4 text file into the MPS grammar language and secondly to transform this grammar into the target MPS language.}

\paragraph{mps-metabnf.}
The mps-metabnf tool~\cite{ref:MPSMETABNF}, implemented by members of the DSLFoundry group, builds upon similar ideas as the ANTLR{\_}MPS project described above.
Authors of this project, too, created an MPS language that can be used to capture ANTLR grammars.
Every imported grammar is represented using this ANTLRv4 MPS language.
A user is then able to adjust the grammar as he wishes, specifying the code layout in the projectional editor.
In the last step, the target MPS language is derived from the adjusted grammar definition.
\PV{Navrhuji odsud dolu smazat.}
Many iterations of this process, involving adjustments of the grammar followed by recreation of the target language, can be performed.

The main limitation of this approach is the need to perform the last step manually, because MPS currently does not provide any support for automated transformation of a grammar captured by the ANTLRv4 MPS language into the actual target language.
\PV{Smazat az sem.}
\PV{A pridat: This step has not yet been implemented, because MPS doesn't offer any mechanisms for automatic transformation.
This means that implementation would require traversing the AST of the grammar and recreating the definition from that.}