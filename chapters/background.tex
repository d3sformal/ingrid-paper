\section{Background}
\label{sect:BACKGROUND}

Here we introduce basic features of the JetBrains MPS platform, and then we briefly describe other projects that aim to add support for general purpose languages into MPS.

\subsection{JetBrains MPS}
\label{sect:MPS}

When using the MPS projectional editor, the developer creates programs by assembling the AST out of predefined building blocks of selected languages.
The definition of a language in MPS dictates where in the AST certain elements can be placed and how they can be nested inside other elements.
Code of any program is actually built from AST nodes, which represent instances of concepts from the languages that the program is written in.
In MPS, a \emph{concept} is a language element, i.e. a building block of a language definition.

A key advantage of projectional editing stems from the separation of abstract and concrete syntax.
While AST provides a complete and precise representation of the code, the way it is displayed on the screen and the way the user edits it are unconstrained.
The editor can take any visual form and shape (i.e., not just textual).
The language author can define multiple alternative visualizations and let the developer choose one that fits best the task at hand.

The definition of an MPS concept (language element) consists of several aspects, where each aspect codifies a different part of the AST nodes' behavior.
The essential aspects are the following: \emph{Structure}, \emph{Editor} and \emph{TextGen}.
Structure represents the abstract syntax (types and hierarchy of AST nodes), Editor defines the concrete syntax (i.e., how the code is visualized and edited) and TextGen specifies how AST nodes are transformed into textual representation.
Other aspects, such as type constraints and model-to-model transformation rules, are not relevant for the contribution of this paper, and therefore we do not describe them here.

Languages are built from the concepts using techniques known from object-oriented programming --- containment, inheritance, and so on.
Therefore, the definition of a language in MPS has an object-oriented and hierarchical nature.

\noindent\textbf{Structure.}
The fundamental aspect of any MPS language is Structure.
It specifies core attributes of a language concept such as the name, inheritance relationships, child concepts (their types and cardinalities), and references to other AST nodes.
Structure also restricts the type of AST nodes that can appear at a particular place in the tree.
For example, one can restrict the condition in the \code{if-then-else} statement to be a boolean expression, and the \code{then}-block to be a list of statements.
Figure~\ref{fig:if_statement_structure} in Appendix~\ref{sect:APX} shows the most important fragments of the Structure aspect for the \code{if-then-else} statement.

\noindent\textbf{Editor.}
The Editor aspect is where the language designer specifies what the projectional representation of a code fragment (an AST) looks like on the screen and how the user interacts with the code.
JetBrains have developed a cellular system that enables placing properties and children of a node (concept) into different cells.
Individual cells of the editor can be styled using a language similar to CSS.
Supported visual characteristics include color and indentation.

\noindent\textbf{BaseLanguage.}
Another important feature of MPS is \emph{BaseLanguage}~\cite{ref:BASELANG}, a clone of Java implemented using the MPS constructs.
BaseLanguage is syntactically almost identical to Java, but it is edited in a projectional editor.
Custom DSLs meant to be generated into Java choose BaseLanguage as their generation target.

\noindent\textbf{TextGen.}
The TextGen aspect specifies how an AST node is translated into plain text representation.
It is typically needed only for the bottom-line base languages.
We provide examples of the Editor and TextGen aspects for the \code{if-else} statement in the Appendix~\ref{sect:APX}.

\subsection{Related Projects with Similar Goals}
\label{sect:RELATED}

We are aware just of a few attempts to create ports of general-purpose languages into IDE tools based on projectional editing.
All of them (described below) were conducted manually.

The BaseLanguage~\cite{ref:BASELANG} is an almost full port of Java, extended with MPS-specific features.
The C language has also been manually tailored for MPS within the mbeddr project~\cite{ref:MBEDDR}.

The following three projects --- PE4MPS, ANTLR{\_}MPS, and mps-metabnf --- aim to provide certain support for GPPLs within the MPS platform.
We describe their main features and limitations.

\noindent\textbf{PE4MPS~\cite{ref:PE4MPS}.}
This project addresses the lack of information about code layout in grammars by using so called PE grammars~\cite{ref:PE}, which is a notation proposed by the same author.
PE grammars extend the syntax of the ANTLR v4 notation by constructs that provide information about the possible layout of AST nodes.
Information about the code layout, extracted from the PE file, is used when generating the projectional editor for every AST node.

The main disadvantage of the PE4MPS approach is that it only shifts the tedious manual work from creating projectional editors in the MPS IDE to writing PE grammars in plain text.
Usage of a standard text editor is inferior (and much more error-prone) with respect to MPS, which has been designed for the purpose of specifying the code layout and provides lot of support to developers.

\noindent\textbf{ANTLR{\_}MPS~\cite{ref:ANTLR2MPS}.}
The author of this project created an MPS language that captures the syntax of ANTLR v4 notation using MPS concepts.
Given the textual grammar of some language as input, the grammar is imported automatically into MPS, taking the form of the ANTLR v4 MPS language.
The next step would be to create a new MPS language from the grammar definition in MPS, but it is not implemented yet.
Another important limitation of this project is the completely missing support for the Editor and TextGen aspects, and it also does not generate any parent-child relationships in the Structure aspect.

\noindent\textbf{mps-metabnf~\cite{ref:MPSMETABNF}.}
This tool builds upon similar ideas as the ANTLR{\_}MPS project.
Every imported grammar is represented using an ANTLR v4 MPS language.
A user is then able to adjust the grammar, specifying the code layout in the projectional editor.
In the last step, the target MPS language is derived from the grammar.

The main limitation of this approach is the need to perform the last step manually, because MPS does not provide any support for automated transformation of a grammar captured by the ANTLR v4 MPS language into the definition of the actual target MPS language.

