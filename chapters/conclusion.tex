\section{Conclusion and Future Work}

The proposed INGRID method is only partially-automated, but nevertheless it reduces the amount of necessary manual work/effort (that the user has to perform) to a great degree.

\todo{Highlight again:} The challenges and problems discussed in this paper, and our solution, are not specific to MPS --- they are common for everyone tackling a similar problem (in the field of DSL and language workbenches).

\todo{Write something.}

There is lot of further follow-up work to be done. Many open problems remain.

Besides using some kind of automated (machine) learning to generate the Editor aspect, we would also like to do XYZ.
Generating nicely designed and user-friendly editors, maybe using artificial intelligence (machine learning) that would evaluate (analyze) existing code in plain text and based on that create the editors.
Moreover, once we would have some information about the layout for generating better projectional editor, we could leverage the same information and use it to improve TextGen.

Follow-up work might look more into the problem of code layout detection, possibly exploring suggested approaches such as the learning approach described in Section XY \todo{Tohle se snad podari aspon castecne implementovat}.
This could lead to better results in editor and TextGen aspect generation which could definitely use some improvement.
Another future endeavors will definitely touch the subject of generating parsers that would enable the user to import existing source code into MPS.
