\section{Conclusion and Future Work}

Our contribution is the \textsc{Ingrid} method for constructing language definitions in MPS from their grammars in the ANTLR notation.
The method is only partially automated, but nevertheless it greatly reduces the amount of manual work that developers of MPS languages have to perform.
Although we discussed all the challenges, details of our solution and general observations just in the context of ANTLR grammars and JetBrains MPS, we believe they are more general --- and in particular relevant to everyone tackling a similar problem in the field of DSLs and language workbenches.

In the future, we would like to add support for generating other aspects of MPS languages and increase the level of automation.
Our primary research goal is to investigate the usage of automated (machine) learning for the purpose of generating editors that support nice and readable code layout.
First, we would have to collect a set of input source code files to be used for training. This step might involve a user study designed in order to identify examples of layout that a majority considers as nice and readable.
The information about code layout, once it is available, will be then leveraged and used also to improve TextGen.
Another subject for future work is automated construction of parser-based tools that could be used for translating source code in plain-text to instances of MPS languages.

