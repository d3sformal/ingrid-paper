\section{Conclusion and Future Work}

Our contribution is the \textsc{Ingrid} method for constructing language definitions in MPS from their grammars in the ANTLR notation.
The method is only partially automated, but nevertheless it greatly reduces the amount of manual work that developers of MPS languages have to perform.
Although we discussed all the challenges, details of our solution and general observations just in the context of ANTLR grammars and JetBrains MPS, we believe they are more general --- and in particular relevant to everyone tackling a similar problem in the field of DSLs and language workbenches.

In the future, we would like to address the remaining open problems (there are many of them) and increase the level of automation.
Our primary research goal is to investigate the usage of machine learning for the purpose of generating editors that support nice and readable code layout.
The information about code layout, once it is available, will be then leveraged and used also to improve TextGen.
Another subject for future work is generating parsers that would allow users to import existing source code into MPS \todo{Vysvetlit podrobneji ze chceme automaticky generovat parsery textovych zdrojaku ktere budou vytvaret interni reprezentace (ASTs) postavene na zaklade MPS concepts pro dany jazyk}.

