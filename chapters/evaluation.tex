\section{Evaluation}
\label{sect:EVAL}

We implemented the proposed INGRID method as an MPS plugin that allows users to generate languages from ANTLR v4 grammars.
While most of the plugin is written in Java, small fragments of BaseLanguage code were needed to call the MPS API, e.g. in order to programmatically create new language concepts and their aspects.
The plugin uses the ANTLR library~\cite{ref:ANTLR} for parsing of ANTLR v4 grammar files, and several MPS libraries that implement the MPS API.
Our complete implementation is available at \url{https://github.com/premun/ingrid}.

For the purpose of evaluation, we applied INGRID to several well-established and widely used languages, including JSON, JavaScript, and C\#.
We used the version of JavaScript that is standardized as ECMAScript 5.1~\cite{ref:ECMASCRIPT51} --- the specification was finalized in June 2011, and it is currently the most frequently adopted version.
MPS projects that contain definitions of all three languages are also released in the public repository (\url{https://github.com/premun/ingrid}).
In the rest of this section, we discuss our experience with application of INGRID to these languages, and then we highlight few general observations.

However, first we must emphasize that MPS languages automatically produced by the INGRID method, as defined in this paper, are not ready-to-use full-fledged MPS counterparts of the original input languages.
The structure of such a generated MPS language fully corresponds to the respective ANTLR grammar, but its other aspects (e.g., the Editor) would have to be manually tweaked (or defined from scratch) by the end user.
INGRID does not yet support advanced features of MPS, such as the aspect responsible for type checking and inference.
For each of the three languages (JSON, JavaScript, and C\#), we provide (1) its MPS definition exactly in the form created by INGRID and (2) the adjusted ANTLR v4 grammar used as input for INGRID, both in the repository that contains also our implementation.

\paragraph{JSON.}
The least amount of manual adjusting after the import into MPS was needed in case of the JSON language~\cite{ref:JSON}, because it is the simplest language from all that we used for our experiments.
Specifically, the first author spent less than 20 minutes in order to get a language that is ready to use.

\paragraph{JavaScript.}
In the case of JavaScript, which is an example of a widely-used complex general purpose language, automated generating of the language definition in MPS from the ANTLR grammar\footnote{https://github.com/antlr/grammars-v4/blob/master/ecmascript/ECMAScript.g4} and subsequent manual adjusting was done in less than one hour.

We are aware of other projects that aim to create a manual port of JavaScript into MPS.
For example, there is ECMAScript4MPS~\cite{ref:ECMA4MPS} developed by the author of the PE4MPS project~\cite{ref:PE4MPS} that we described in Section~\ref{sect:RELATED}.

The main advantage of INGRID over ECMAScript4MPS is that, despite its current limitations, INGRID fully automatically produces the definition of JavaScript in MPS that needs just minor adjustments to be really useful.
INGRID achieved a very good result especially in the case of language structure, concept aliases, and support for auto-completion.
The Structure aspect generated by INGRID is very similar to that of the ECMAScript4MPS project, which was created manually in a large number of hours.

\paragraph{C\#.}
Another very complex programming language that we used for experiments is C\#.
Before we could run INGRID, we had to manually adjust the ANTLR grammar of C\# in order to ensure that INGRID produces a reasonably useful MPS language.
Roughly one hour of manual effort was needed for this step.
Nevertheless, additional modifications of the grammar and selected aspects of the MPS language (Editor, TextGen) are still needed to get an optimal result.

The generated definition of C\$ in MPS is quite large, involving more than 800 concepts, and therefore building of the language takes a rather long time.

\paragraph{Other languages.}
We also tried to apply INGRID on few other languages, such as Python and Ruby.
Results are mixed because ANTLR grammars of these languages are written in a style that is not fully compatible with INGRID.
For example, INGRID cannot handle cyclic rules (which appear in the Python grammar) and just prints a warning on the screen.
Significant modifications of the grammar would be necessary.
We plan to address this limitation in the future.

\paragraph{General observations.}
The main overall benefit of the INGRID method is partial automation.
Most of the languages discussed above, which we tried to create in MPS by using INGRID, are quite complex regarding their structure and syntax variety.
Therefore, completely manual definition would be a very time-consuming and error-prone process.
Fully automated generation of the Structure aspect is where INGRID spares the user from many hours of tedious and sometimes quite challenging work.

On the other hand, our experiments with complex languages show that, in the case of the Editor and Textgen aspects, manual adjustment (e.g., adding line breaks and indentation) is a very effective approach that takes only a short time --- in particular, much less time than we expected.
The first author spent between 20 and 60 minutes of work on each language, when adjusting the result of automated generation into a more useful and readable form.
MPS IDE provides good support for efficient tweaking of the code layout in Editor and TextGen, allowing users to reach optimal results very fast.
All of this justifies some of our decisions that we have described in Section~\ref{sect:EDITORDEF} and Section~\ref{sect:TEXTGENDEF}, proving that our approach is quite practical.
Despite that, we also tried to design some automated heuristics during our work on the INGRID project, but so far all yield rather average results especially when compared to what human users can achieve efficiently.

